\documentclass[11pt,a4paper]{article}
\usepackage{amsmath}
\usepackage{amssymb}
\usepackage[capitalise,noabbrev,nameinlink]{cleveref}
\usepackage{extarrows}
\usepackage{float}
\usepackage[margin=2.5cm]{geometry}
\usepackage[unicode=true,pdftex,pdfa]{hyperref}
\usepackage[utf8]{inputenc}
\usepackage{latexsym}
\usepackage{mathpazo} % nice fonts
\usepackage{mathtools}
\usepackage{microtype}
\usepackage[colon]{natbib}
%%
%% Package `semantic` can be used for writing inference rules.
%%
\usepackage{semantic}
\usepackage{slashed}
\usepackage{stmaryrd}
\usepackage[colorinlistoftodos,prependcaption,textsize=tiny]{todonotes}

\hypersetup{
  pdftitle={Specification of the Blockchain Layer},
  breaklinks=true,
  bookmarks=true,
  colorlinks=false,
  linkcolor={blue},
  citecolor={blue},
  urlcolor={blue},
  linkbordercolor={white},
  citebordercolor={white},
  urlbordercolor={white}
}
\floatstyle{boxed}
\restylefloat{figure}

%% Setup for the semantic package
\setpremisesspace{20pt}

\DeclareMathOperator{\dom}{dom}
\DeclareMathOperator{\range}{range}

%%
%% TODO: we should package this
%%
\newcommand{\powerset}[1]{\mathbb{P}~#1}
\newcommand{\restrictdom}{\lhd}
\newcommand{\subtractdom}{\mathbin{\slashed{\restrictdom}}}
\newcommand{\restrictrange}{\rhd}
\newcommand{\union}{\cup}
\newcommand{\unionoverride}{\mathbin{\underrightarrow\cup}}
\newcommand{\uniondistinct}{\uplus}
\newcommand{\var}[1]{\mathit{#1}}
\newcommand{\fun}[1]{\mathsf{#1}}
\newcommand{\type}[1]{\mathsf{#1}}
\newcommand{\signed}[2]{\llbracket #1 \rrbracket_{#2}}
\newcommand{\size}[1]{\left| #1 \right|}
\newcommand{\trans}[2]{\xlongrightarrow[\textsc{#1}]{#2}}
\newcommand{\seqof}[1]{#1^{*}}
\newcommand{\nextdef}{\\[1em]}

%%
%% Types
%%
\newcommand{\Tx}{\type{Tx}}
\newcommand{\UTxO}{\type{UTxO}}
\newcommand{\Value}{\type{Value}}
% Adding witnesses
\newcommand{\TxIn}{\type{TxIn}}
\newcommand{\TxOut}{\type{TxOut}}
\newcommand{\VKey}{\type{VKey}}
\newcommand{\Sig}{\type{Sig}}
\newcommand{\Data}{\type{Data}}
\newcommand{\Hash}{\type{Hash}}

%%
%% Functions
%%
\newcommand{\txins}[1]{\fun{txins}\ \var{#1}}
\newcommand{\txouts}[1]{\fun{txouts}\ \var{#1}}
\newcommand{\values}[1]{\fun{values}\ #1}
\newcommand{\balance}[1]{\fun{balance}\ \var{#1}}
% Adding witnesses...
\newcommand{\inputs}[1]{\fun{inputs}\ \var{#1}}
\newcommand{\witnesses}[1]{\fun{witnesses}\ \var{#1}}
\newcommand{\verify}[3]{\fun{verify} ~ #1 ~ #2 ~ #3}
\newcommand{\serialised}[1]{\llbracket \var{#1} \rrbracket}
\newcommand{\addr}[1]{\fun{addr}\ \var{#1}}
\newcommand{\hash}[1]{\fun{hash}\ \var{#1}}
\newcommand{\inout}[3]{\var{#1}\mapsto_{#2}\var{#3}}
% Adding ledgers...
\newcommand{\utxo}[1]{\fun{utxo}\ #1}

\begin{document}

\title{Specification of the Blockchain Layer}

\author{Marko Dimjašević}

\date{September 28, 2018}

\maketitle

\begin{abstract}
This documents defines a semantics for operations on a blockchain.
\end{abstract}

\tableofcontents
\listoffigures

\section{Introduction}
\label{sec:introduction}

\section{Preliminaries}
\label{sec:preliminaries}

\section{Basic definitions}
\label{sec:basic-definitions}

\section{Auxiliary definitions}
\label{sec:auxil-defin}

\section{State transitions for Blockchain}
\label{sec:state-trans-chain}

\subsection{Properties}
\label{sec:chain-properties}

\begin{figure}[h]
  \emph{Abstract types}
  %
  \begin{align*}
    & \TxIn & \text{transaction input}\\
    & \TxOut & \text{transaction output}\\
    & \VKey & \text{verification key}\\
    & \Sig  & \text{signature}\\
    & \Data  & \text{data}\\
    & \Hash  & \text{hash}\\
  \end{align*}
  \emph{Abstract functions}
  %
  \begin{align*}
    & \witnesses{} \in \Tx \mapsto \powerset{(\VKey \times \Sig)}
    & \text{witnesses of a transaction}\\
    %
    & \inputs{} \in \Tx \mapsto \TxIn
    & \text{transaction inputs}\\
    %
    & \verify{}{}{} \in \VKey \times \Data \times \Sig
    & \text{verification relation}\\
    %
    & \addr{} \in \TxOut \mapsto \Hash
    & \text{output address} \\
    %
    & \hash{} \in \VKey \mapsto \Hash
    & \text{hash function}
  \end{align*}
  \caption{Definitions associated with the blockchain transition system}
  \label{fig:state-trans-abstract}
\end{figure}

\end{document}

%%% Local Variables:
%%% mode: latex
%%% TeX-master: t
%%% End:
