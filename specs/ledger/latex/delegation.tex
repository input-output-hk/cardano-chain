\newcommand{\DCert}{\type{DCert}}
\newcommand{\DState}{\type{DState}}

An agent owning a key that can sign new blocks can delegate its signing rights
to another key by means of \textit{delegation certificates}. These certificates
are included in the ledger, and therefore also included in the body of the
blocks in the blockchain.

In the blockchain protocol only a certain number of keys can sign blocks, and
the verifying part of these keys\footnote{Also known as public keys.} are
maintained in the genesis block. One important restriction on delegation is
that only the keys in the genesis block can delegate to other keys. However, at
the ledger level we do not know which are these keys, and thus this is a
restriction to be enforced at the blockchain level. In this formalization we
only care about establishing, whether $\var{vk}_s$ delegated its rights to
$\var{vk}_d$. To keep track of this, we use a map from keys to keys.

The rule for delegation is presented in
Figure~\ref{fig:rules:delegation}. It states that if $\var{c}$ is a valid
delegation certificate from key $\var{vk}_s$ to key $\var{vk}_d$, then the
delegation map $d$ is updated to contain the key mapping
$\var{vk}_s \mapsto \var{vk}_d$. The symbol $\unionoverride$ denotes
union-override, and is defined in Figure~\ref{fig:defs:delegation}. Another
condition that is checked in the delegation rule is that each genesis key can
issue at most one delegation certificate per epoch. To check this, the states
of the transitions system maintain a map from epochs to set of keys that were
delegated on a given epoch. Finally, we only allow delegations for the next
epoch. This is what allow us to clean up the maps of keys delegated per epoch.
Variables to the left of the $\vdash$ (turnstile) symbol are the
\textit{environment} of a rule, which remains constant. This environment (the
current epoch in this case) would be set by the (chain) rules that use
Rule~\ref{eq:delegation}.

\begin{figure}
  \emph{Abstract types}
  %
  \begin{equation*}
    \begin{array}{r@{~\in~}lr}
      c & \DCert & \text{delegation certificate}\\
      vk_g & \VKeyGen & \text{genesis verification key}\\
      e & \Epoch & \text{epoch}\\
    \end{array}
  \end{equation*}
  \emph{Constraints}
  \begin{align*}
    \VKeyGen \subseteq \VKey
  \end{align*}
  \emph{Abstract functions}
  \begin{align*}
    & \fun{dwho} \in \DCert \mapsto (\VKeyGen \times \VKey) & \text{who delegates to who in the certificate}\\
    & \fun{depoch} \in \DCert \mapsto \Epoch & \text{certificate epoch}
  \end{align*}
  \caption{Delegation definitions}
  \label{fig:defs:delegation}
\end{figure}

\begin{figure}
  \begin{align*}
    & \unionoverride \in (A \mapsto B) \to (A \mapsto B) \to (A \mapsto B)
    & \text{union override}\\
    & d_0 \unionoverride d_1 = d_1 \cup (\dom d_1 \subtractdom d_0)
  \end{align*}
  \caption{Functions used in delegation rules}
  \label{fig:funcs:delegation}
\end{figure}

\begin{figure}
  \emph{Delegation states}
  \begin{align*}
    & \DState
      = \left(
        \begin{array}{r@{~\in~}lr}
          \var{dmap} & \VKeyGen \mapsto \VKey & \text{delegation map}\\
          \var{ekeys} & \Epoch \mapsto \powerset{\VKeyGen} & \text{keys delegated per epoch}
        \end{array}\right)
  \end{align*}
  \emph{Delegation transitions}
  \begin{equation*}
    \_ \vdash \_ \trans{deleg}{\_} \_ \in
      \powerset (\Epoch \times \DState \times \DCert \times \DState)
    \end{equation*}
  \caption{Delegation transition-system types}
  \label{fig:ts-types:delegation}
\end{figure}

\begin{figure}
  \begin{equation}\label{eq:delegation}
    \inference[Delegation]
    {\dwho{c} = (vk_s, vk_d) & \depoch{c} = e_c & \var{vk_s} \notin \var{ekeys}~e_c & \var{cepoch} + 1 = e_c\\
      \var{ekeys'} = \{ (e, \var{vks})  \mid (e, \var{vks}) \in \var{ekeys},~  \var{cepoch} \leq e \}
    }
    {
      \var{cepoch} \vdash
      \left(
      \begin{array}{r}
        \var{dmap}\\
        \var{ekeys}
      \end{array}
      \right)
      \trans{deleg}{c}
      \left(
      \begin{array}{lcl}
        \var{dmap} & \unionoverride & \{\var{vk_s} \mapsto \var{vk_d}\}\\
        \var{ekeys'} & \unionoverride & \{e_c \mapsto (\var{ekeys'}~e_c \cup \{\var{vk_s}\})\}
      \end{array}
      \right)
    }
  \end{equation}
  \caption{Delegation inference rules}
  \label{fig:rules:delegation}
\end{figure}

\subsection{Witnesses}
\label{sec:delegation-witnesses}

The rule for certificate witnesses is given in
Figure~\ref{fig:rules:delegationw}. The new definitions introduced in this rule
are given in Figure~\ref{fig:defs:delegationw}.

\begin{figure}
  \emph{Primitive types}
  \begin{equation*}
    \begin{array}{r@{~\in~}lr}
      \var{epoch} & \Epoch & \text{epoch}
    \end{array}
  \end{equation*}
  \emph{Abstract functions}
  \begin{equation*}
    \begin{array}{r@{~\in~}lr}
      \fun{dbody} & \DCert \to (\VKey \times \Epoch)
      & \text{body of the delegation certificate}\\
      \fun{dwit} & \DCert \to (\VKeyGen \times \Sig)
      & \text{witness for the delegation certificate}
    \end{array}
  \end{equation*}
  \caption{Delegation witnesses definitions}
  \label{fig:defs:delegationw}
\end{figure}

\begin{figure}
  \emph{Delegation with witness rule}
  \begin{equation}
    \label{eq:deleg-witnesses}
    \inference[Deleg-wit]
    { \dwit{c} = (\var{vk_s}, \sigma)
      & \var{cepoch} \vdash
      {\left(
        \begin{array}{r}
          \var{dmap}\\
          \var{ekeys}
        \end{array}
      \right)}
      \trans{deleg}{c}
      {\left(
      \begin{array}{l}
          \var{dmap}\\
          \var{ekeys}
      \end{array}
      \right)}
      \\ ~ \\
      \verify{vk_s}{\serialised{\dbody{c}}}{\sigma}
    }
    {\var{cepoch} \vdash
      {\left(
        \begin{array}{r}
          \var{dmap}\\
          \var{ekeys}
        \end{array}
      \right)}
      \trans{delegw}{c}
      {\left(
      \begin{array}{l}
          \var{dmap}\\
          \var{ekeys}
      \end{array}
      \right)}}
  \end{equation}
  \caption{Delegation witnesses inference rules}
  \label{fig:rules:delegationw}
\end{figure}
