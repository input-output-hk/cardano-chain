\newcommand{\DSEnv}{\type{DSEnv}}
\newcommand{\DSState}{\type{DSState}}
\newcommand{\DCert}{\type{DCert}}
\newcommand{\DState}{\type{DState}}
\newcommand{\DEState}{\type{DEState}}

An agent owning a key that can sign new blocks can delegate its signing rights
to another key by means of \textit{delegation certificates}. These certificates
are included in the ledger, and therefore also included in the body of the
blocks in the blockchain.

There are several restrictions on a certificate posted on the blockchain:
\begin{enumerate}
\item Only genesis keys can delegate.
\item Certificates must be properly signed by the delegator.
\item Any given key can delegate at most once per-epoch.
\item The epochs in certificate cannot refer to past epochs. This mechanism
  prevents replay attacks.
\item Two certificates for the same key issued in the same (current) slot
  cannot contain different epochs.
\item Certificates do not become active immediately, but they require a certain
  number of slots till they become stable in all the nodes.
\end{enumerate}
These conditions are formalized in \cref{fig:rules:delegation-scheduling}.
Rule~\ref{eq:rule:delegation-scheduling} determines when a certificate can
become ``scheduled''. The definitions used in this rules are presented in
\cref{fig:defs:delegation-scheduling}, and the types of the system induced by
$\trans{sdeleg}{\wcard}$ are presented in
\cref{fig:ts-types:delegation-scheduling}.

\begin{figure}
  \emph{Abstract types}
  \begin{equation*}
    \begin{array}{r@{~\in~}lr}
      c & \DCert & \text{delegation certificate}\\
      \var{vk_g} & \VKeyGen & \text{genesis verification key}\\
    \end{array}
  \end{equation*}

  \emph{Derived types}
  \begin{equation*}
    \begin{array}{r@{~\in~}l@{\qquad=\qquad}r@{~\in~}lr}
      \var{e} & \Epoch & n & \mathbb{N} & \text{epoch}\\
      \var{s} & \Slot & s & \mathbb{N} & \text{slot}
    \end{array}
  \end{equation*}

  \emph{Constraints}
  \begin{align*}
    \VKeyGen \subseteq \VKey
  \end{align*}

  \emph{Abstract functions}
  \begin{equation*}
    \begin{array}{r@{~\in~}lr}
      \fun{dbody} & \DCert \to (\VKey \times \Epoch)
      & \text{body of the delegation certificate}\\
      \fun{dwit} & \DCert \to (\VKeyGen \times \Sig)
      & \text{witness for the delegation certificate}\\
      \fun{dwho} & \DCert \mapsto (\VKeyGen \times \VKey)
      & \text{who delegates to who in the certificate}\\
      \fun{depoch} & \DCert \mapsto \Epoch
      & \text{certificate epoch}
    \end{array}
  \end{equation*}
  \caption{Delegation scheduling definitions}
  \label{fig:defs:delegation-scheduling}
\end{figure}

\begin{figure}
  \emph{Delegation scheduling environments}
  \begin{equation*}
    \DSEnv =
    \left(
      \begin{array}{r@{~\in~}lr}
        \mathcal{K} & \powerset{\VKeyGen} & \text{allowed delegators}\\
        \var{e} & \Epoch & \text{epoch}\\
        \var{s} & \Slot & \text{slot}\\
        \var{d} & \Slot & \text{certificate liveness parameter}
      \end{array}
    \right)
  \end{equation*}

  \emph{Delegation scheduling states}
  \begin{equation*}
    \DSState
    = \left(
      \begin{array}{r@{~\in~}lr}
        \var{sds} & \seqof{(\Slot \times (\VKeyGen \times \VKey))} & \text{scheduled delegations}\\
        \var{eks} & \powerset{(\Epoch \times \VKeyGen)} & \text{key-epoch delegations}
      \end{array}
    \right)
  \end{equation*}
  \caption{Delegation scheduling transition-system types}
  \label{fig:ts-types:delegation-scheduling}
\end{figure}

\begin{figure}
  \begin{equation}
    \label{eq:rule:delegation-scheduling}
    \inference
    {
      \dwit{c} = (\var{vk_s},~ \sigma)
      & \verify{vk_s}{\serialised{\dbody{c}}}{\sigma} & vk_s \in \mathcal{K}\\ ~ \\
      \dwho{c} = (\var{vk_s},~ \var{vk_d}) & \depoch{c} = e_d
      & (e_d,~ \var{vk_s}) \notin \var{eks} & e < e_d \\ ~ \\
      (s + d,~ (\var{vk_s},~ \wcard)) \notin \var{sds}\\
    }
    {
      {\begin{array}{l}
       \mathcal{K}\\ 
        e\\
        s\\
        d
      \end{array}}
      \vdash
      {
        \left(
          \begin{array}{l}
            \var{sds}\\
            \var{eks}
          \end{array}
        \right)
      }
      \trans{sdeleg}{c}
      {
        \left(
          \begin{array}{l}
            \var{sds'}; (s + d,~ (\var{vk_s},~ \var{vk_d}))\\
            \var{eks'} \cup \{(e_d,~ \var{vk_s})\}
          \end{array}
        \right)
      }
    }
  \end{equation}
  \caption{Delegation scheduling rules}
  \label{fig:rules:delegation-scheduling}
\end{figure}

Once a scheduled certificate becomes active
(see~\cref{sec:delegation-interface-rules}), the delegation map is changed by
it. \cref{fig:rules:delegation} models such change.

\begin{figure}
  \begin{align*}
    & \unionoverride \in (A \mapsto B) \to (A \mapsto B) \to (A \mapsto B)
    & \text{union override}\\
    & d_0 \unionoverride d_1 = d_1 \cup (\dom d_1 \subtractdom d_0)
  \end{align*}
  \caption{Functions used in delegation rules}
  \label{fig:funcs:delegation}
\end{figure}

\begin{figure}
  \emph{Delegation states}
  \begin{align*}
    & \DState
      = \left(
        \begin{array}{r@{~\in~}lr}
          \var{dmap} & \VKeyGen \mapsto \VKey & \text{delegation map}
        \end{array}\right)
  \end{align*}
  \emph{Delegation transitions}
  \begin{equation*}
    \_ \vdash \_ \trans{deleg}{\_} \_ \in
      \powerset (\DState \times \DCert \times \DState)
    \end{equation*}
  \caption{Delegation transition-system types}
  \label{fig:ts-types:delegation}
\end{figure}

\begin{figure}
  \begin{equation}\label{eq:delegation}
    \inference[Delegation]
    {\dwho{c} = (vk_s, vk_d)
    }
    {
      \left(
      \begin{array}{r}
        \var{dmap}
      \end{array}
      \right)
      \trans{deleg}{c}
      \left(
      \begin{array}{lcl}
        \var{dmap} & \unionoverride & \{\var{vk_s} \mapsto \var{vk_d}\}
      \end{array}
      \right)
    }
  \end{equation}
  \caption{Delegation inference rules}
  \label{fig:rules:delegation}
\end{figure}
