\documentclass[11pt,a4paper]{article}
\usepackage[margin=2.5cm]{geometry}
\usepackage{microtype}
\usepackage{amsmath}
\usepackage{amssymb}
\usepackage{latexsym}
\usepackage{mathtools}
\usepackage{stmaryrd}
\usepackage{extarrows}
\usepackage{slashed}
\usepackage[colon]{natbib}
\usepackage{todonotes}
\usepackage[unicode=true,pdftex,pdfa]{hyperref}
\usepackage[capitalise,noabbrev,nameinlink]{cleveref}
\usepackage{float}
\floatstyle{boxed}
\restylefloat{figure}

%%
%% Package `semantic` can be used for writing inference rules.
%%
\usepackage{semantic}
%% Setup for the semantic package
\setpremisesspace{20pt}

\DeclareMathOperator{\dom}{dom}
\DeclareMathOperator{\range}{range}

%%
%% TODO: we should package this
%%
\newcommand{\powerset}[1]{\mathbb{P}~#1}
\newcommand{\restrictdom}{\lhd}
\newcommand{\subtractdom}{\mathbin{\slashed{\restrictdom}}}
\newcommand{\restrictrange}{\rhd}
\newcommand{\union}{\cup}
\newcommand{\unionoverride}{\mathbin{\underrightarrow\cup}}
\newcommand{\uniondistinct}{\uplus}
\newcommand{\var}[1]{\mathit{#1}}
\newcommand{\fun}[1]{\mathsf{#1}}
\newcommand{\type}[1]{\mathsf{#1}}
\newcommand{\serialised}[1]{\llbracket #1 \rrbracket}
\newcommand{\signed}[2]{\llbracket #1 \rrbracket_{#2}}
\newcommand{\verified}[3]{\mathcal{V}_{#1}\llbracket #2 \rrbracket_{#3}}
\newcommand{\size}[1]{\left| #1 \right|}
\newcommand{\transitionarrow}[2]{\xlongrightarrow[\textsc{#1}]{#2}}
\newcommand{\listOf}[1]{#1^{*}}
\newcommand{\nextdef}{\\[1em]}

%%
%% Types
%%
\newcommand{\Tx}{\type{Tx}}
\newcommand{\UTxO}{\type{UTxO}}
\newcommand{\Coin}{\type{Coin}}

%%
%% Functions
%%
\newcommand{\txins}[1]{\text{txins}\ \var{#1}}
\newcommand{\txouts}[1]{\text{txouts}\ \var{#1}}
\newcommand{\balance}[1]{\text{balance}\ \var{#1}}
\newcommand{\unspent}[1]{\text{unspent}\ \var{#1}}
\begin{document}

\title{Simplified Formal Specification of a UTxO Ledger}

\author{}

\date{September 24, 2018}

\begin{abstract}
Bleh.
\end{abstract}

\section{Introduction}
\label{sec:introduction}

This specification models the \textit{conditions} required for the extension of
a ledger, i.e. a list of transactions. The following aspects are part of such
conditions:

\begin{description}
\item[Balances] relationship between the total value of input and outputs
  in a new transaction, and the unspent outputs.
\item[Witnesses] cryptographic entities needed to validate the expenditure
  of a transaction input.
\item[Heavyweight-delegation] transfer of staking rights.
\item[Update validation] voting mechanism which captures the identification of
  the voters, and the participants that can post update proposals.
\end{description}

\section{Preliminaries}\label{sec:preliminaries}

TODO: explain  $\powerset{\type{X}}$, $\listOf{\type{X}}$, etc.


\section{Basic definitions}
\label{sec:basic-definitions}

\section{Auxiliary definitions}
\label{sec:auxil-defin}

\section{State transitions for UTxO}
\label{sec:state-trans-utxo-1}

The states of the UTxO transition system, along with their associated functions
and types are defined in Figure~\ref{fig:state-trans-utxo-defs}.
\begin{figure*}[h]
  \emph{UTxO States}
  %
  \begin{equation*}
    \var{u} \in \powerset{\UTxO}
  \end{equation*}
  %
  \emph{UTxO Transitions}
  \begin{equation*}
    \var{u} \transitionarrow{utxo}{tx} \var{u'}
    \in (\powerset{UTxO}) \times \Tx \times (\powerset{\UTxO})
  \end{equation*}  
  %
  \emph{Abstract Types}
  \begin{align*}
    & \type{UTxO} & \text{unspent transaction outputs}
    \nextdef
    & \type{Tx} & \text{transactions}
  \end{align*}
  \emph{Abstract Functions}
  \begin{align*}
    & \txins{} \in \Tx \mapsto \UTxO & \text{transaction spent inputs}
    \nextdef
    & \txouts{} \in \Tx \mapsto \UTxO & \text{transaction outputs}
    \nextdef
    & \balance{} \in \UTxO \mapsto \Coin & \text{unspent balance}
  \end{align*}
  \emph{Constraints}
  \begin{align*}
    & (\Coin, \leq)~\text{is a total order} & \text{total order on }\Coin
  \end{align*}
  \caption{Definitions associated to the UTxO transition system}
  \label{fig:state-trans-utxo-defs}
\end{figure*}

The transition rules for unspent outputs are presented in Figure~\ref{fig:state-trans-utxo}.

\begin{figure}[h]

  \centering
  \begin{equation}\label{eq:utxo-base}
    \inference[UTxO-base]
    {}
    {\emptyset}
  \end{equation}
  
  \begin{equation}\label{eq:utxo-inductive}
    \inference[UTxO-inductive]
    { \txins{tx} \subseteq \mathit{u}
      & \balance{(\txouts{tx})} \leq \balance{(\txins{tx})}
    }
    {u \transitionarrow{utxo}{tx} (u \setminus \txins{tx}) \cup \unspent{tx}}
  \end{equation}
  \caption{State transitions for UTxO transactions}
  \label{fig:state-trans-utxo}
\end{figure}

Rule~\ref{eq:utxo-base} simply states that the empty set is a valid set of
unspent outputs.

Rule~\ref{eq:utxo-inductive} specifies under which conditions a transaction can
be applied to a set of unspent outputs, and how the set of unspent output changes
with a transaction:
\begin{itemize}
\item The set spent inputs in the transaction, must be in the set of unspent
  outputs.
\item The balance of the unspent outputs in a transaction (i.e. the total
  amount paid in a transaction) must be less or equal than the spent
  inputs.
\end{itemize}

\section{State transitions for basic UTxO transactions}
\label{sec:state-trans-basic}

\section{State transitions for UTxO transactions with witnesses}
\label{sec:state-trans-utxo}

\end{document}
