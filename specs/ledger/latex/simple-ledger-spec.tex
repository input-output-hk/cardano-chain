\documentclass[11pt,a4paper]{article}
\usepackage[margin=2.5cm]{geometry}
\usepackage{microtype}
\usepackage{mathpazo} % nice fonts
\usepackage{amsmath}
\usepackage{amssymb}
\usepackage{latexsym}
\usepackage{mathtools}
\usepackage{stmaryrd}
\usepackage{extarrows}
\usepackage{slashed}
\usepackage[colon]{natbib}
\usepackage[unicode=true,pdftex,pdfa]{hyperref}
\usepackage{xcolor}
\usepackage[capitalise,noabbrev,nameinlink]{cleveref}
\hypersetup{
  pdftitle={A Simplified Formal Specification of a UTxO Ledger},
  breaklinks=true,
  bookmarks=true,
  colorlinks=false,
  linkcolor={blue},
  citecolor={blue},
  urlcolor={blue},
  linkbordercolor={white},
  citebordercolor={white},
  urlbordercolor={white}
}
\usepackage{float}
\floatstyle{boxed}
\restylefloat{figure}
% For notes containing warnings, questions, etc.
\usepackage[tikz]{bclogo}
\newenvironment{question}
  {\begin{bclogo}[logo=\bcquestion, couleur=orange!10, arrondi=0.2]{ QUESTION}}
  {\end{bclogo}}
\newenvironment{todo}
  {\begin{bclogo}[logo=\bcoutil, couleur=red!5, couleurBarre=red, arrondi=0.2]{ TODO}}
  {\end{bclogo}}
%%
%% Package `semantic` can be used for writing inference rules.
%%
\usepackage{semantic}
%% Setup for the semantic package
\setpremisesspace{20pt}

\DeclareMathOperator{\dom}{dom}
\DeclareMathOperator{\range}{range}

%%
%% TODO: we should package this
%%
\newcommand{\powerset}[1]{\mathbb{P}~#1}
\newcommand{\restrictdom}{\lhd}
\newcommand{\subtractdom}{\mathbin{\slashed{\restrictdom}}}
\newcommand{\restrictrange}{\rhd}
\newcommand{\union}{\cup}
\newcommand{\unionoverride}{\mathbin{\underrightarrow\cup}}
\newcommand{\uniondistinct}{\uplus}
\newcommand{\var}[1]{\mathit{#1}}
\newcommand{\fun}[1]{\mathsf{#1}}
\newcommand{\type}[1]{\mathsf{#1}}
\newcommand{\signed}[2]{\llbracket #1 \rrbracket_{#2}}
\newcommand{\size}[1]{\left| #1 \right|}
\newcommand{\trans}[2]{\xlongrightarrow[\textsc{#1}]{#2}}
\newcommand{\seqof}[1]{#1^{*}}
\newcommand{\nextdef}{\\[1em]}
\newcommand{\where}{~ ~ \mathbf{where}~ ~ }

%%
%% Types
%%
\newcommand{\Tx}{\type{Tx}}
\newcommand{\Ix}{\type{Ix}}
\newcommand{\TxId}{\type{TxId}}
\newcommand{\Addr}{\type{Addr}}
\newcommand{\UTxO}{\type{UTxO}}
\newcommand{\Value}{\type{Value}}
\newcommand{\Coin}{\type{Coin}}
% Adding witnesses
\newcommand{\TxIn}{\type{TxIn}}
\newcommand{\TxOut}{\type{TxOut}}
\newcommand{\VKey}{\type{VKey}}
\newcommand{\Sig}{\type{Sig}}
\newcommand{\Data}{\type{Data}}
% Adding delegation
\newcommand{\DCert}{\type{DCert}}
\newcommand{\DState}{\type{DState}}
\newcommand{\Epoch}{\type{Epoch}}

%%
%% Functions
%%
\newcommand{\txins}[1]{\fun{txins}~ \var{#1}}
\newcommand{\txid}[1]{\fun{txid}~ \var{#1}}
\newcommand{\txouts}[1]{\fun{txouts}~ \var{#1}}
\newcommand{\values}[1]{\fun{values}~ #1}
\newcommand{\balance}[1]{\fun{balance}~ \var{#1}}
%% UTxO witnesses
\newcommand{\inputs}[1]{\fun{inputs}~ \var{#1}}
\newcommand{\wits}[1]{\fun{wits}~ \var{#1}}
\newcommand{\verify}[3]{\fun{verify} ~ #1 ~ #2 ~ #3}
\newcommand{\serialised}[1]{\llbracket \var{#1} \rrbracket}
\newcommand{\addr}[2]{\fun{addr}~ \var{#1}~ \var{#2}}
\newcommand{\hash}[1]{\fun{hash}~ \var{#1}}
\newcommand{\txbody}[1]{\fun{txbody}~ \var{#1}}
% wildcard parameter
\newcommand{\wcard}[0]{\underline{\phantom{a}}}
%% Adding ledgers...
\newcommand{\utxo}[1]{\fun{utxo}~ #1}
%% Delegation
\newcommand{\delegatesName}{\fun{delegates}}
\newcommand{\delegates}[3]{\delegatesName~#1~#2~#3}
\newcommand{\dwho}[1]{\fun{dwho}~\var{#1}}
\newcommand{\depoch}[1]{\fun{depoch}~\var{#1}}
%% Delegation witnesses
\newcommand{\dbody}[1]{\fun{dbody}~\var{#1}}
\newcommand{\dwit}[1]{\fun{dwit}~\var{#1}}


\begin{document}

\title{A Simplified Formal Specification of a UTxO Ledger}

\author{}

\date{September 24, 2018}

\maketitle

\begin{abstract}
This documents defines the rules for extending a ledger with transactions. It
is intended to serve as the specification for random generators of transactions
which adhere to the rules presented here.
\end{abstract}

\tableofcontents
\listoffigures

\section{Introduction}
\label{sec:introduction}

This specification models the \textit{conditions} required for the extension of
a ledger, which is modeled here as a list of transactions. The following
aspects are part of such conditions:

\begin{description}
\item[Preservation of value] relationship between the total value of input and outputs
  in a new transaction, and the unspent outputs.
\item[Witnesses] cryptographic entities needed to validate the spending
  of a transaction input.
\item[Delegation] heavyweight delegation certificates.
\end{description}

Aspects that will not be modeled since they are not part of the legacy-free
implementation:
\begin{description}
\item[Stake] outputs do not have a transfer of stake associated to them.
\item[Update validation] voting mechanism which captures the identification of
  the voters, and the participants that can post update proposals.
\end{description}
\section{Preliminaries}\label{sec:preliminaries}

\begin{description}
\item[Powerset] Given a set $\type{X}$, $\powerset{\type{X}}$ is the set of all
  the subsets of $X$.
\item[Sequences] Given a set $\type{X}$, $\seqof{\type{X}}$ is the set of
  sequences having elements taken from $\type{X}$. The empty sequence is
  denoted by $\epsilon$, and given a sequence $\Lambda$, $\Lambda; \type{x}$ is
  the sequence that results from appending $\type{x} \in \type{X}$ to
  $\Lambda$.
\end{description}

\section{Basic definitions}
\label{sec:basic-definitions}

\section{Auxiliary definitions}
\label{sec:auxil-defin}

\section{State transitions for UTxO}
\label{sec:state-trans-utxo-1}

The states of the UTxO transition system, along with their types are defined in
Figure~\ref{fig:state-trans-utxo-defs}. Functions on these types are defined in
Figure~\ref{fig:utxo-auxiliary-ops}.

\begin{figure*}
  \emph{Abstract types}
  \begin{equation*}
    \begin{array}{r@{~\in~}lr}
      \var{tx} & \Tx & \text{transaction}\\
    \end{array}
  \end{equation*}
  \emph{Primitive types}
  %
  \begin{equation*}
    \begin{array}{r@{~\in~}lr}
      \var{txid} & \TxId & \text{transaction id}\\
      %
      ix & \Ix & \text{index}\\
      %
      \var{addr} & \Addr & \text{address}\\
      %
      c & \Coin & \text{currency value}
    \end{array}
  \end{equation*}
  \emph{Derived types}
  %
  \begin{equation*}
    \begin{array}{r@{~\in~}l@{\qquad=\qquad}r@{~\in~}lr}
      \var{txin}
      & \TxIn
      & (\var{txid}, \var{ix})
      & \TxId \times \Ix
      & \text{transaction input}
      \\
      \var{txout}
      & \type{TxOut}
      & (\var{addr}, c)
      & \Addr \times \Coin
      & \text{transaction output}
      \\
      \var{utxo}
      & \UTxO
      & \var{txin} \mapsto \var{txout}
      & \TxIn \mapsto \TxOut
      & \text{unspent tx outputs}
    \end{array}
  \end{equation*}
  \emph{UTxO States}
  %
  \begin{equation*}
    \var{utxo} \in \UTxO
  \end{equation*}
  %
  \emph{UTxO Transitions}
  \begin{equation*}
    \var{\_} \trans{utxo}{\_} \var{\_}
    \subseteq \powerset (\UTxO \times \Tx \times \UTxO)
  \end{equation*}
  %
  \emph{Abstract Functions}
  \begin{align*}
    & \txid{} \in \Tx \mapsto \TxId & \text{compute transaction id}\\
    %
    & \fun{txbody} \in \Tx \mapsto \powerset{\TxIn} \times (\Ix \mapsto \TxOut)
    & \text{transaction body}
  \end{align*}
  \caption{Definitions associated to the UTxO transition system}
  \label{fig:state-trans-utxo-defs}
\end{figure*}

\begin{figure}
  \begin{align*}
    & \fun{txins} \in \Tx \to \powerset{\TxIn}
    & \text{transaction inputs} \\
    & \txins{tx} = \var{inputs} \where \txbody{tx} = (\var{inputs}, ~\wcard)
    \nextdef
    & \fun{txouts} \in \Tx \to \UTxO
    & \text{transaction outputs as UTxO} \\
    & \fun{txouts} ~ \var{tx} =
      \left\{ (\fun{txid} ~ \var{tx}, \var{ix}) \mapsto \var{txout} ~
      \middle| \begin{array}{l@{~}c@{~}l}
                 (\_, \var{outputs}) & = & \txbody{tx} \\
                 \var{ix} \mapsto \var{txout} & \in & \var{outputs}
               \end{array}
      \right\}
    \nextdef
    & \fun{balance} \in \UTxO \to \Coin
    & \text{UTxO balance} \\
    & \fun{balance} ~ utxo = \sum_{(~\wcard ~ \mapsto (\wcard, ~c)) \in \var{utxo}} c
  \end{align*}

  \begin{align*}
    \var{ins} \restrictdom \var{utxo}
    & = \{ i \mapsto o \mid i \mapsto o \in \var{utxo}, ~ i \in \var{ins} \}
    & \text{domain restriction}
    \\
    \var{ins} \subtractdom \var{utxo}
    & = \{ i \mapsto o \mid i \mapsto o \in \var{utxo}, ~ i \notin \var{ins} \}
    & \text{domain exclusion}
    \\
    \var{utxo} \restrictrange \var{outs}
    & = \{ i \mapsto o \mid i \mapsto o \in \var{utxo}, ~ o \in \var{outs} \}
    & \text{range restriction}
  \end{align*}
  \caption{UTxO operations}
  \label{fig:utxo-auxiliary-ops}
\end{figure}

The transition rules for unspent outputs are presented in
Figure~\ref{fig:state-trans-utxo}. Rule~\ref{eq:utxo-base} simply states that
we can start with any arbitrary value for unspent outputs ($\UTxO$).
Rule~\ref{eq:utxo-inductive} specifies under which conditions a transaction can
be applied to a set of unspent outputs, and how the set of unspent output changes
with a transaction:
\begin{itemize}
\item The set spent inputs in the transaction, must be in the set of unspent
  outputs.
\item The balance of the unspent outputs in a transaction (i.e. the total
  amount paid in a transaction) must be equal less than the amount of spent
  inputs.
\item If the above conditions hold, then the new state will not have the inputs
  spent in transaction $\var{tx}$ and it will have the new outputs in
  $\var{tx}$.
\end{itemize}

\begin{figure}

  \centering
  \begin{equation}\label{eq:utxo-base}
    \inference[UTxO-base]
    {}
    {\var{utxo}}
  \end{equation}

  \begin{equation}\label{eq:utxo-inductive}
    \inference[UTxO-inductive]
    { \txins{tx} \subseteq \var{utxo}
      & \balance{(\txouts{tx})} \leq \balance{(\txins{tx} \restrictdom \var{utxo})}
    }
    {\var{utxo} \trans{utxo}{tx}
      (\txins{tx} \subtractdom \var{utxo}) \cup \txouts{tx}
    }
  \end{equation}
  \caption{UTxO inference rules}
  \label{fig:state-trans-utxo}
\end{figure}

\subsection{Properties}
\label{sec:utxo-properties}


\begin{todo}
  Can we prove properties of the transition system of this section? For
  instance I'd like to formalize ``double spending'' and prove that these rules
  prevent it. Do we want it?
\end{todo}

\subsection{Witnesses}
\label{sec:witnesses}

The rules for witnesses are presented in Figure~\ref{fig:rules-utxo-witnesses}.
Note that the transition $u \trans{}{tx} u'$ is any arbitrary transition on
$\UTxO \times \Tx \times \UTxO$, which means that we're defining a family of
operational rules. This is what allow us to compose operational rules as
needed. The definitions used in Rule~\ref{eq:utxo-witness-inductive} are given
in Figure~\ref{fig:state-trans-utxo-witnesses-defs}.

\begin{figure}
  \begin{equation}
    \label{eq:utxo-witness-inductive}
    \inference[UTxO-wit]
    { \var{utxo} \trans{}{tx} \var{utxo'}\\
      & \forall i \in \txins{tx} \cdot
           \exists (\var{vk}, \sigma) \in \wits{\var{tx}} \cdot
              \verify{vk}{\serialised{\txbody{tx}}}{\sigma}  \wedge \addr{i}{utxo} = \hash{vk}
    }
    {\var{utxo} \trans{utxow}{tx} \var{utxo'}}
  \end{equation}
  \caption{UTxO with witnesses inference rules}
  \label{fig:rules-utxo-witnesses}
\end{figure}

\begin{figure}
  \emph{Abstract types}
  %
  \begin{align*}
    & \VKey & \text{verification key}\\
    & \Sig  & \text{signature}\\
    & \Data  & \text{data}\\
  \end{align*}
  \emph{Abstract functions}
  %
  \begin{align*}
    & \fun{wits} \in \Tx \mapsto \powerset{(\VKey \times \Sig)}
    & \text{witnesses of a transaction}\\
    %
    & \fun{verify} \in \VKey \times \Data \times \Sig
    & \text{verification relation}\\
    %
    & \hash{} \in \VKey \mapsto \Addr
    & \text{hash function} \\
  \end{align*}
  \emph{Functions}
  %
  \begin{align*}
    & \addr{}{} \in \TxIn \mapsto \UTxO \mapsto \Addr\\
    & \addr{i}{utxo} = \var{hk} \where (hk, \wcard) = \var{utxo}~i
  \end{align*}
  \caption{Definitions associated to the UTxO transition system with witnesses}
  \label{fig:state-trans-utxo-witnesses-defs}
\end{figure}

\section{Delegation}
\label{sec:delegation}

An agent owning a key that can sign new blocks can delegate its signing rights
to another key by means of \textit{delegation certificates}. These certificates
are included in the ledger, and therefore also included in the body of the
blocks in the blockchain.

In the blockchain protocol only a certain number of keys can sign blocks, and
the signing key part of these keys are maintained in the genesis block. One
important restriction on delegation is that only the keys in the genesis
block can delegate to other keys. However, at the ledger level we do not know
which are these keys, and thus this is a restriction to be enforced at the
blockchain level. In this formalization we only care about establishing,
whether $\var{vk}_s$ delegated its rights to $\var{vk}_d$. To keep track of
this, we use a map from keys to keys.

The rule for delegation is presented in
Figure~\ref{fig:state-trans-delegation}. It states that if $\var{c}$ is a valid
delegation certificate from key $\var{vk}_s$ to key $\var{vk}_d$, then the
delegation map $d$ is updated to contain the key mapping
$\var{vk}_s \mapsto \var{vk}_d$. The symbol $\unionoverride$ denotes
union-override, and is defined in Figure~\ref{fig:delegation-defs}.

\begin{figure}

  \begin{equation}\label{eq:delegation}
    \inference[Delegation]
    {\dwho{c} = (vk_s, vk_d) & \depoch{c} = e & \var{vk_s} \notin \var{depoch}~e}
    {
      \left(
      \begin{array}{r}
        \var{dmap}\\
        \var{depoch}
      \end{array}
      \right)
      \trans{delegates}{c}
      \left(
      \begin{array}{lcl}
        \var{dmap} & \unionoverride & \{\var{vk_s} \mapsto \var{vk_d}\}\\
        \var{depoch} & \unionoverride & \{e \mapsto (\var{depoch}~e \cup \{\var{vk_s}\})\}
      \end{array}
      \right)
    }
  \end{equation}
  \caption{Delegation inference rules}
  \label{fig:state-trans-delegation}
\end{figure}

\begin{figure}
  \emph{Abstract types}
  %
  \begin{equation*}
    \begin{array}{r@{~\in~}lr}
      c & \DCert  & \text{delegation certificates}\\
    \end{array}
  \end{equation*}
  \emph{Delegation states}
  \begin{equation*}
    \DState =
    \left(\begin{array}{r@{~\in~}lr}
      \var{dmap} & \VKey \mapsto \VKey & \text{delegation map}\\
      \var{depoch} & \Epoch \mapsto \powerset{\VKey} & \text{delegation state}
    \end{array}\right)
  \end{equation*}
  \emph{Delegation transition}
  \begin{equation*}
    \_ \trans{delegates}{\_} \_ \in
      \powerset (\DState \times \DCert \times \DState)
  \end{equation*}
  \emph{Abstract functions}
  \begin{align*}
    & \fun{dwho} \in \DCert \mapsto (\VKey \times \VKey) & \text{who delegates to who in the certificate}
  \end{align*}
  \emph{Functions}
  \begin{align*}
    (d_0 \unionoverride d_1)~k=
    \begin{cases}
      d_1~k & k \in \dom d_1\\
      d_0~k & \text{otherwise}\\
    \end{cases}
  \end{align*}
  \caption{Delegation definitions}
  \label{fig:delegation-defs}
\end{figure}

\subsection{Witnesses}
\label{sec:delegation-witnesses}

The rule for certificate witnesses is given in
Figure~\ref{fig:deleg-witnesses}. The new definitions introduced in this rule
are given in Figure~\ref{fig:delegation-witnesses-defs}.

\begin{figure}
  \begin{equation}
    \label{eq:deleg-witnesses}
    \inference[deleg-wit]
    { \dwit{c} = (\var{vk_s}, \sigma) & \var{dmap} \trans{}{c} \var{dmap'} \\
      \verify{vk_s}{\serialised{\dbody{c}}}{\sigma}
    }
    {\var{dmap} \trans{delegw}{c} \var{dmap'}}
  \end{equation}
  \caption{Delegation witnesses inference rules}
  \label{fig:deleg-witnesses}
\end{figure}

\begin{figure}
  \emph{Primitive types}
  \begin{equation*}
    \begin{array}{r@{~\in~}lr}
      \var{epoch} & \Epoch & \text{epoch}
    \end{array}
  \end{equation*}
    \emph{Abstract functions}
  \begin{align*}
      & \fun{dbody} \in \DCert \mapsto \VKey \times \Epoch & \text{body of the delegation certificate}\\
      & \fun{dwit} \in \DCert \mapsto (\VKey \times \Sig) & \text{witness for the delegation certificate}
  \end{align*}
  \caption{Delegation witnesses definitions}
  \label{fig:delegation-witnesses-defs}
\end{figure}

\end{document}
