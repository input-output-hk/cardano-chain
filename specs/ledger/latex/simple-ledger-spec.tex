\documentclass[11pt,a4paper]{article}
\usepackage[margin=2.5cm]{geometry}
\usepackage{microtype}
\usepackage{mathpazo} % nice fonts
\usepackage{amsmath}
\usepackage{amssymb}
\usepackage{latexsym}
\usepackage{mathtools}
\usepackage{stmaryrd}
\usepackage{extarrows}
\usepackage{slashed}
\usepackage[colon]{natbib}
\usepackage[colorinlistoftodos,prependcaption,textsize=tiny]{todonotes}
\usepackage[unicode=true,pdftex,pdfa]{hyperref}
\usepackage[capitalise,noabbrev,nameinlink]{cleveref}
\hypersetup{
  pdftitle={A Simplified Formal Specification of a UTxO Ledger},
  breaklinks=true,
  bookmarks=true,
  colorlinks=false,
  linkcolor={blue},
  citecolor={blue},
  urlcolor={blue},
  linkbordercolor={white},
  citebordercolor={white},
  urlbordercolor={white}
}
\usepackage{float}
\floatstyle{boxed}
\restylefloat{figure}

%%
%% Package `semantic` can be used for writing inference rules.
%%
\usepackage{semantic}
%% Setup for the semantic package
\setpremisesspace{20pt}

\DeclareMathOperator{\dom}{dom}
\DeclareMathOperator{\range}{range}

%%
%% TODO: we should package this
%%
\newcommand{\powerset}[1]{\mathbb{P}~#1}
\newcommand{\restrictdom}{\lhd}
\newcommand{\subtractdom}{\mathbin{\slashed{\restrictdom}}}
\newcommand{\restrictrange}{\rhd}
\newcommand{\union}{\cup}
\newcommand{\unionoverride}{\mathbin{\underrightarrow\cup}}
\newcommand{\uniondistinct}{\uplus}
\newcommand{\var}[1]{\mathit{#1}}
\newcommand{\fun}[1]{\mathsf{#1}}
\newcommand{\type}[1]{\mathsf{#1}}
\newcommand{\signed}[2]{\llbracket #1 \rrbracket_{#2}}
\newcommand{\size}[1]{\left| #1 \right|}
\newcommand{\trans}[2]{\xlongrightarrow[\textsc{#1}]{#2}}
\newcommand{\seqof}[1]{#1^{*}}
\newcommand{\nextdef}{\\[1em]}

%%
%% Types
%%
\newcommand{\Tx}{\type{Tx}}
\newcommand{\UTxO}{\type{UTxO}}
\newcommand{\Value}{\type{Value}}
% Adding witnesses
\newcommand{\TxIn}{\type{TxIn}}
\newcommand{\TxOut}{\type{TxOut}}
\newcommand{\VKey}{\type{VKey}}
\newcommand{\Sig}{\type{Sig}}
\newcommand{\Data}{\type{Data}}
\newcommand{\Hash}{\type{Hash}}

%%
%% Functions
%%
\newcommand{\txins}[1]{\fun{txins}\ \var{#1}}
\newcommand{\txouts}[1]{\fun{txouts}\ \var{#1}}
\newcommand{\values}[1]{\fun{values}\ #1}
\newcommand{\balance}[1]{\fun{balance}\ \var{#1}}
% Adding witnesses...
\newcommand{\inputs}[1]{\fun{inputs}\ \var{#1}}
\newcommand{\witnesses}[1]{\fun{witnesses}\ \var{#1}}
\newcommand{\verify}[3]{\fun{verify} ~ #1 ~ #2 ~ #3}
\newcommand{\serialised}[1]{\llbracket \var{#1} \rrbracket}
\newcommand{\addr}[1]{\fun{addr}\ \var{#1}}
\newcommand{\hash}[1]{\fun{hash}\ \var{#1}}
\newcommand{\inout}[3]{\var{#1}\mapsto_{#2}\var{#3}}
% Adding ledgers...
\newcommand{\utxo}[1]{\fun{utxo}\ #1}

\begin{document}

\title{A Simplified Formal Specification of a UTxO Ledger}

\author{}

\date{September 24, 2018}

\maketitle

\begin{abstract}
This documents defines the rules for extending a ledger with transactions. It
is intended to serve as the specification for random generators of transactions
which adhere to the rules presented here.
\end{abstract}

\tableofcontents
\listoffigures

\section{Introduction}
\label{sec:introduction}

This specification models the \textit{conditions} required for the extension of
a ledger, which is modeled here as a list of transactions. The following
aspects are part of such conditions:

\begin{description}
\item[Preservation of value] relationship between the total value of input and outputs
  in a new transaction, and the unspent outputs.
\item[Witnesses] cryptographic entities needed to validate the spending
  of a transaction input.
\item[Delegation] heavyweight delegation certificates.  
\end{description}

Aspects that will not be modeled since they are not part of the legacy-free
implementation:
\begin{description}
\item[Stake] outputs do not have a transfer of stake associated to them.
\item[Update validation] voting mechanism which captures the identification of
  the voters, and the participants that can post update proposals.
\end{description}
\section{Preliminaries}\label{sec:preliminaries}

\begin{description}
\item[Powerset] Given a set $\type{X}$, $\powerset{\type{X}}$ is the set of all
  the subsets of $X$.
\item[Sequences] Given a set $\type{X}$, $\seqof{\type{X}}$ is the set of
  sequences having elements taken from $\type{X}$. The empty sequence is
  denoted by $\epsilon$, and given a sequence $\Lambda$, $\Lambda; \type{x}$ is
  the sequence that results from appending $\type{x} \in \type{X}$ to
  $\Lambda$.
\end{description}

\section{Basic definitions}
\label{sec:basic-definitions}

\section{Auxiliary definitions}
\label{sec:auxil-defin}

\section{State transitions for UTxO}
\label{sec:state-trans-utxo-1}

The states of the UTxO transition system, along with their associated functions
and types are defined in Figure~\ref{fig:state-trans-utxo-defs}. In particular,
we leave the definition of $\UTxO$ and $\Value$ abstract, and place constraints
on the choice of these types.

Regarding $\UTxO$, we require that $(\UTxO, \cup, \emptyset)$ is a commutative
\textit{monoid} where $\cup$ is the addition operation and $\emptyset$ the
neutral element, with a relation $\subseteq$ and a binary operator $\setminus$
(called \textit{monus}) where:

\begin{itemize}
\item $\var{u} \subseteq \var{v}$ iff there exists another element $\var{w}$
  such that $\var{u} \cup \var{w} = \var{v}$.
\item $\var{u} \setminus \var{v}$ is the \textbf{unique} smallest element
  $\var{w}$ such that $\var{u} \subseteq \var{v} + \var{w}$.
\end{itemize}

Note that the notation chosen suggests that $\UTxO$ is a set, however this does not
necessarily have to be the case.

As for $\Value$ the only requirement in these rules is that it forms a
\textit{monoid} w.r.t. a binary operation $+$, and a neutral element $0$. Given
a sequence of elements $\var{vs} = v_0, \ldots, v_{n-1}$, such that $v_i \in \Value$, we
write $\sum_{v \in \var{vs}} v$ for the sum of all elements in $\var{vs}$ using
the $+$ operation.

\begin{figure*}[h]
  \emph{UTxO States}
  %
  \begin{equation*}
    \var{u} \in \UTxO
  \end{equation*}
  %
  \emph{UTxO Transitions}
  \begin{equation*}
    \var{u} \trans{utxo}{tx} \var{u'}
    \in \UTxO \times \Tx \times \UTxO
  \end{equation*}
  %
  \emph{Abstract Types}
  \begin{align*}
    & \type{UTxO} & \text{unspent transaction outputs}
    \nextdef
    & \type{Tx} & \text{transactions}
  \end{align*}
  \emph{Abstract Functions}
  \begin{align*}
    & \txins{} \in \Tx \mapsto \UTxO & \text{transaction inputs (to be spent)}
    \nextdef
    & \txouts{} \in \Tx \mapsto \UTxO & \text{transaction outputs}
    \nextdef
    & \values{} \in \UTxO \mapsto \seqof{\Value} & \text{values in an UTxO}
  \end{align*}
  \emph{Constraints}
  \begin{align*}
    & (\Value, +) & \text{monoid}\\
    & (\UTxO, \cup, \emptyset, \setminus, \subseteq) & \text{commutative monoid with monus}
  \end{align*}
  \caption{Definitions associated to the UTxO transition system}
  \label{fig:state-trans-utxo-defs}
\end{figure*}

The transition rules for unspent outputs are presented in
Figure~\ref{fig:state-trans-utxo}. We make use of function $\balance{}$, which is
defined below:

\begin{align*}
  & \balance{} \in \UTxO \mapsto \Value\\
  & \balance{u} = \sum_{v \in \values{u}} v
\end{align*}

\begin{figure}[h]

  \centering
  \begin{equation}\label{eq:utxo-base}
    \inference[UTxO-base]
    {}
    {\var{u}}
  \end{equation}

  \begin{equation}\label{eq:utxo-inductive}
    \inference[UTxO-inductive]
    { \txins{tx} \subseteq \mathit{u}
      & \balance{(\txouts{tx})} = \balance{(\txins{tx})}
    }
    {u \trans{utxo}{tx} (u \setminus \txins{tx}) \cup \txouts{tx}}
  \end{equation}
  \caption{UTxO inference rules}
  \label{fig:state-trans-utxo}
\end{figure}

Rule~\ref{eq:utxo-base} simply states that we can start with any arbitrary
value for unspent outputs ($\UTxO$).

Rule~\ref{eq:utxo-inductive} specifies under which conditions a transaction can
be applied to a set of unspent outputs, and how the set of unspent output changes
with a transaction:
\begin{itemize}
\item The set spent inputs in the transaction, must be in the set of unspent
  outputs.
\item The balance of the unspent outputs in a transaction (i.e. the total
  amount paid in a transaction) must be equal than the spent inputs. If fees
  and forged value are present in the model, then we assume fees to be part of
  the outputs, and forged value to be part of the inputs.
\item If the above conditions hold, then the new state will not have the inputs
  spent in transaction $\var{tx}$ and it will have the new outputs in
  $\var{tx}$.
\end{itemize}

\subsection{Properties}
\label{sec:utxo-properties}

\todo{Can we prove properties of the transition system of this section? For
instance I'd like to formalize ``double spending'' and prove that these rules
prevent it. Do we want it?}

\section{Witnesses}
\label{sec:witnesses}

\begin{figure}[h]
  \emph{Abstract types}
  %
  \begin{align*}
    & \TxIn & \text{transaction input}\\
    & \TxOut & \text{transaction output}\\
    & \VKey & \text{verification key}\\
    & \Sig  & \text{signature}\\
    & \Data  & \text{data}\\
    & \Hash  & \text{hash}\\
  \end{align*}
  \emph{Abstract functions}
  %
  \begin{align*}
    & \witnesses{} \in \Tx \mapsto \powerset{(\VKey \times \Sig)}
    & \text{witnesses of a transaction}\\
    %
    & \inout{\_}{\_}{\_} \in \TxIn \times \UTxO \times \TxOut
    & \text{input maps to output in utxo}\\
    %
    & \inputs{} \in \Tx \mapsto \TxIn
    & \text{transaction inputs}\\
    %
    & \verify{}{}{} \in \VKey \times \Data \times \Sig
    & \text{verification relation}\\
    %
    & \addr{} \in \TxOut \mapsto \Hash
    & \text{output address} \\
    %
    & \hash{} \in \VKey \mapsto \Hash
    & \text{hash function}
  \end{align*}
  \caption{Definitions associated to the UTxO transition system with witnesses}
  \label{fig:state-trans-utxo-witnesses-defs}
\end{figure}

The rules for witnesses are presented in Figure~\ref{fig:rules-utxo-witnesses}.
Note that the transition $u \trans{}{tx} u'$ is any arbitrary transition on
$\UTxO \times \Tx \times \UTxO$, which means that we're defining a family of
operational rules. This is what allow us to compose operational rules as
needed.

\begin{figure}[h]
  \begin{equation}
    \label{eq:utxo-witness-inductive}
    \inference[UTxO-wit-inductive]
    {wits = \witnesses{tx} \wedge \inout{i}{u}{o}
      & u \trans{}{tx} u'\\
      & \forall i \in \inputs{tx} ~ ~ \exists (\var{vk}, \sigma) \in wits ~ . ~
      \verify{vk}{\serialised{tx}}{\sigma}  \wedge \addr{o} = \hash{vk}
    }
    {u \trans{utxow}{tx} u'}
  \end{equation}
  \caption{UTxO with witnesses inference rules}
  \label{fig:rules-utxo-witnesses}
\end{figure}

\section{Delegation}
\label{sec:delegation}

We model heavyweight delegation, since heavyweight delegation certificates are included in
the ledger. Lightweight delegation has to be modeled in the chain model, and it
outside the scope of this specification.

This is related to delegation of rights to sign blocks.

Requirements on the delegation certificate:
- Certificate must be signed.

Do we need to keep into account the seven bootstrap keys from the genesis
block.

Why do we need to include the certificate in the chain? Why not including it
when signing a block? This is related to the question about the fundamental
difference between heavyweight and lightweight delegation certificates.

\section{State transitions for ledgers}
\label{sec:state-trans-ledg}

To define the state transition system for ledgers we need the definitions
shown in Figure~\ref{fig:ledger-rules-defs}.

\begin{figure}[h]
  \emph{Abstract functions}
  %
  \begin{align*}
    & \utxo{} \in \seqof{Tx} \mapsto \UTxO
      & \text{unspent transactions outputs of a ledger}
  \end{align*}
  \caption{Definitions used in ledger rules}
  \label{fig:ledger-rules-defs}
\end{figure}

Note that the rules presented in Figure~\ref{fig:ledger-rules} do no specify
anything about the UTxO transition in the premise. We only need that the
$\utxo{}$ function can be computed for a list of transactions.

\begin{figure}[h]
  \begin{equation}
    \label{eq:ledger-base}
    \inference[Ledger-base]
    {}
    {\epsilon}
  \end{equation}

  \begin{equation}
    \label{eq:ledger-indutive}
    \inference[Ledger-inductive]
    {\utxo{\Lambda} = u & u \trans{}{tx} u' }
    {\Lambda \trans{ledger}{tx}{\Lambda; \var{tx}}}
  \end{equation}
  \caption{Ledger inference rules}
  \label{fig:ledger-rules}
\end{figure}

\end{document}
