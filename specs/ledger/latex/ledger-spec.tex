\documentclass[11pt,a4paper]{article}
\usepackage[margin=2.5cm]{geometry}
\usepackage{iohk}
\usepackage{microtype}
\usepackage{mathpazo} % nice fonts
\usepackage{amsmath}
\usepackage{amssymb}
\usepackage{latexsym}
\usepackage{mathtools}
\usepackage{stmaryrd}
\usepackage{extarrows}
\usepackage{slashed}
\usepackage[colon]{natbib}
\usepackage[unicode=true,pdftex,pdfa]{hyperref}
\usepackage{xcolor}
\usepackage[capitalise,noabbrev,nameinlink]{cleveref}
\usepackage{float}
\floatstyle{boxed}
\restylefloat{figure}

%%
%% Package `semantic` can be used for writing inference rules.
%%
\usepackage{semantic}
%% Setup for the semantic package
\setpremisesspace{20pt}

%%
%% Types
%%
\newcommand{\Bool}{\type{Bool}}
\newcommand{\Tx}{\type{Tx}}
\newcommand{\Ix}{\type{Ix}}
\newcommand{\TxId}{\type{TxId}}
\newcommand{\Addr}{\type{Addr}}
\newcommand{\UTxO}{\type{UTxO}}
\newcommand{\Value}{\type{Value}}
\newcommand{\Coin}{\type{Coin}}
\newcommand{\PrtclConsts}{\type{PrtclConsts}}
%% Adding witnesses
\newcommand{\TxIn}{\type{TxIn}}
\newcommand{\TxOut}{\type{TxOut}}
\newcommand{\VKey}{\type{VKey}}
\newcommand{\SKey}{\type{SKey}}
\newcommand{\Hash}{\type{Hash}}
\newcommand{\SkVk}{\type{SkVk}}
\newcommand{\Sig}{\type{Sig}}
\newcommand{\Data}{\type{Data}}
%% Adding delegation
\newcommand{\Epoch}{\type{Epoch}}
\newcommand{\VKeyGen}{\type{VKeyGen}}
%% Blockchain
\newcommand{\Gkeys}{\var{G_{keys}}}
\newcommand{\Block}{\type{Block}}
\newcommand{\SlotId}{\type{SlotId}}
\newcommand{\CEEnv}{\type{CEEnv}}
\newcommand{\CEState}{\type{CEState}}
\newcommand{\BDEnv}{\type{BDEnv}}
\newcommand{\BDState}{\type{BDState}}
\newcommand{\Slot}{\type{Slot}}

%%
%% Functions
%%
\newcommand{\txins}[1]{\fun{txins}~ \var{#1}}
\newcommand{\txid}[1]{\fun{txid}~ \var{#1}}
\newcommand{\txouts}[1]{\fun{txouts}~ \var{#1}}
\newcommand{\values}[1]{\fun{values}~ #1}
\newcommand{\balance}[1]{\fun{balance}~ \var{#1}}
%% UTxO witnesses
\newcommand{\inputs}[1]{\fun{inputs}~ \var{#1}}
\newcommand{\wits}[1]{\fun{wits}~ \var{#1}}
\newcommand{\verify}[3]{\fun{verify} ~ #1 ~ #2 ~ #3}
\newcommand{\sign}[2]{\fun{sign} ~ #1 ~ #2}
\newcommand{\serialised}[1]{\llbracket \var{#1} \rrbracket}
\newcommand{\addr}[1]{\fun{addr}~ \var{#1}}
\newcommand{\hash}[1]{\fun{hash}~ \var{#1}}
\newcommand{\txbody}[1]{\fun{txbody}~ \var{#1}}
\newcommand{\txfee}[1]{\fun{txfee}~ \var{#1}}
\newcommand{\minfee}[2]{\fun{minfee}~ \var{#1}~ \var{#2}}
% wildcard parameter
\newcommand{\wcard}[0]{\underline{\phantom{a}}}
%% Adding ledgers...
\newcommand{\utxo}[1]{\fun{utxo}~ #1}
%% Delegation
\newcommand{\delegatesName}{\fun{delegates}}
\newcommand{\delegates}[3]{\delegatesName~#1~#2~#3}
\newcommand{\dwho}[1]{\fun{dwho}~\var{#1}}
\newcommand{\depoch}[1]{\fun{depoch}~\var{#1}}
%% Delegation witnesses
\newcommand{\dbody}[1]{\fun{dbody}~\var{#1}}
\newcommand{\dwit}[1]{\fun{dwit}~\var{#1}}
%% Blockchain
\newcommand{\bwit}[1]{\fun{bwit}~\var{#1}}
\newcommand{\bslot}[1]{\fun{bslot}~\var{#1}}
\newcommand{\bbody}[1]{\fun{bbody}~\var{#1}}
\newcommand{\bdlgs}[1]{\fun{bdlgs}~\var{#1}}

\begin{document}

\hypersetup{
  pdftitle={A Simplified Formal Specification of a UTxO Ledger},
  breaklinks=true,
  bookmarks=true,
  colorlinks=false,
  linkcolor={blue},
  citecolor={blue},
  urlcolor={blue},
  linkbordercolor={white},
  citebordercolor={white},
  urlbordercolor={white}
}

\title{A Simplified Formal Specification of a UTxO Ledger}

\author{}

\date{September 24, 2018}

\maketitle

\begin{abstract}
This documents defines the rules for extending a ledger with transactions. It
is intended to serve as the specification for random generators of transactions
which adhere to the rules presented here.
\end{abstract}


\tableofcontents
\listoffigures

\section{Introduction}
\label{sec:introduction}
This specification models the \textit{conditions} that the different parts of a
transaction have to fulfill so that they can extend a ledger, which is
represented here as a list of transactions. In particular, we model the
following aspects:

\begin{description}
\item[Preservation of value] relationship between the total value of input and
  outputs in a new transaction, and the unspent outputs.
\item[Witnesses] authentication of parts of the transaction data by means of
  cryptographic entities (such as signatures and private keys) contained in
  these transactions.
\item[Delegation] validity of delegation certificates, which delegate
  block-signing rights.
\item[Update validation] voting mechanism which captures the identification of
  the voters, and the participants that can post update proposals.
\end{description}

The following aspects will not be modeled (since they are not part of the legacy-free
implementation):
\begin{description}
\item[Stake] staking rights associated to an addresses.
\end{description}


\section{Notation}\label{sec:notation}

\begin{description}
\item[Powerset] Given a set $\type{X}$, $\powerset{\type{X}}$ is the set of all
  the subsets of $X$.
\item[Sequences] Given a set $\type{X}$, $\seqof{\type{X}}$ is the set of
  sequences having elements taken from $\type{X}$. The empty sequence is
  denoted by $\epsilon$, and given a sequence $\Lambda$, $\Lambda; \type{x}$ is
  the sequence that results from appending $\type{x} \in \type{X}$ to
  $\Lambda$.
\item[Functions] $A \to B$ denotes a \textbf{total function} from $A$ to $B$.
  Given a function $f$ we write $f~a$ for the application of $f$ to argument
  $a$.
\item[Fibre] Given a function $f: A \to B$ and $b\in B$, we write
  $f^{-1}~b$ for the \textbf{fibre} of $f$ at $b$, which is defined by
  $\{a \mid\ f a =  b\}$.
\item[Maps and partial functions] $A \mapsto B$ denotes a \textbf{partial
    function} from $A$ to $B$, which can be seen as a map (dictionary) with
  keys in $A$ and values in $B$. Given a map $m \in A \mapsto B$, notation
  $a \mapsto b \in m$ is equivalent to $m~ a = b$.
\end{description}

\section{Cryptographic primitives}
\label{sec:crypto-primitives}

Figure~\ref{fig:crypto-defs} introduces the cryptographic abstractions used in
this document.

\begin{figure}
  \emph{Abstract types}
  %
  \begin{equation*}
    \begin{array}{r@{~\in~}lr}
      \var{vk} & \SKey & \text{signing key}\\
      \var{vk} & \VKey & \text{verifying key}\\
      \var{hk} & \Hash & \text{hash of a key}\\
      \sigma & \Sig  & \text{signature}\\
      \var{d} & \Data  & \text{data}\\
    \end{array}
  \end{equation*}
  \emph{Derived types}
  \begin{equation*}
    \begin{array}{r@{~\in~}lr}
      (sk, vk) & \SkVk & \text{signing-verifying key pairs}
    \end{array}
  \end{equation*}
  \emph{Abstract functions}
  %
  \begin{equation*}
    \begin{array}{r@{~\in~}lr}
      \hash{} & \VKey \to \Hash
      & \text{hash function} \\
      %
      \fun{verify} & \VKey \times \Data \times \Sig
      & \text{verification relation}\\
    \end{array}
  \end{equation*}
  \emph{Constraints}
  \begin{align*}
    & \forall (sk, vk) \in \SkVk,~ m \in \Data,~ \sigma \in \Sig \cdot
      \verify{vk}{m}{\sigma} \iff \sign{sk}{m} = \sigma
  \end{align*}
  \emph{Notation for serialized and verified data}
  \begin{align*}
    & \serialised{x} & \text{serialised representation of } x\\
    & \mathcal{V}_{\var{vk}}{\serialised{m}}_{\sigma} = \verify{vk}{m}{\sigma}
      & \text{shorthand notation for } \fun{verify}
  \end{align*}
  \caption{Cryptographic definitions}
  \label{fig:crypto-defs}
\end{figure}

\section{Serialization}
\label{sec:serialization}

\begin{todo}
  Discuss here serialization and
  \href{https://iohk.myjetbrains.com/youtrack/issue/CDEC-628}{composable
    serialization}
\end{todo}

\section{UTxO}
\label{sec:state-trans-utxo-1}

The transition rules for unspent outputs are presented in
Figure~\ref{fig:rules:utxo}. The states of the UTxO transition system,
along with their types are defined in Figure~\ref{fig:defs:utxo}.
Functions on these types are defined in Figure~\ref{fig:derived-defs:utxo}.

\begin{figure*}
  \emph{Primitive types}
  %
  \begin{equation*}
    \begin{array}{r@{~\in~}lr}
      \var{txid} & \TxId & \text{transaction id}\\
      %
      ix & \Ix & \text{index}\\
      %
      \var{addr} & \Addr & \text{address}\\
      %
      c & \Coin & \text{currency value}\\
      %
      pc & \PrtclConsts & \text{protocol constants}
    \end{array}
  \end{equation*}
  \emph{Derived types}
  %
  \begin{equation*}
    \begin{array}{r@{~\in~}l@{\qquad=\qquad}r@{~\in~}lr}
      \var{txin}
      & \TxIn
      & (\var{txid}, \var{ix})
      & \TxId \times \Ix
      & \text{transaction input}
      \\
      \var{txout}
      & \type{TxOut}
      & (\var{addr}, c)
      & \Addr \times \Coin
      & \text{transaction output}
      \\
      \var{utxo}
      & \UTxO
      & \var{txin} \mapsto \var{txout}
      & \TxIn \mapsto \TxOut
      & \text{unspent tx outputs}
    \end{array}
  \end{equation*}
  %
  \emph{Abstract types}
  \begin{equation*}
    \begin{array}{r@{~\in~}lr}
      \var{tx} & \Tx & \text{transaction}\\
    \end{array}
  \end{equation*}
  %
  \emph{Abstract Functions}
  \begin{equation*}
    \begin{array}{r@{~\in~}lr}
      \txid{} & \Tx \to \TxId & \text{compute transaction id}\\
      %
      \fun{txbody} & \Tx \to \powerset{\TxIn} \times (\Ix \mapsto \TxOut)
                                  & \text{transaction body}\\
      %
      \fun{txfee} & \Tx \to \Coin & \text{transaction fee}\\
      %
      \fun{minfee} & \PrtclConsts \to \Tx \to \Coin & \text{minimum fee}
    \end{array}
  \end{equation*}
  \caption{Definitions used in the UTxO transition system}
  \label{fig:defs:utxo}
\end{figure*}

\begin{figure}
  \begin{align*}
    & \fun{txins} \in \Tx \to \powerset{\TxIn}
    & \text{transaction inputs} \\
    & \txins{tx} = \var{inputs} \where \txbody{tx} = (\var{inputs}, ~\wcard)
    \nextdef
    & \fun{txouts} \in \Tx \to \UTxO
    & \text{transaction outputs as UTxO} \\
    & \fun{txouts} ~ \var{tx} =
      \left\{ (\fun{txid} ~ \var{tx}, \var{ix}) \mapsto \var{txout} ~
      \middle| \begin{array}{l@{~}c@{~}l}
                 (\_, \var{outputs}) & = & \txbody{tx} \\
                 \var{ix} \mapsto \var{txout} & \in & \var{outputs}
               \end{array}
      \right\}
    \nextdef
    & \fun{balance} \in \UTxO \to \Coin
    & \text{UTxO balance} \\
    & \fun{balance} ~ utxo = \sum_{(~\wcard ~ \mapsto (\wcard, ~c)) \in \var{utxo}} c
  \end{align*}

  \begin{align*}
    \var{ins} \restrictdom \var{utxo}
    & = \{ i \mapsto o \mid i \mapsto o \in \var{utxo}, ~ i \in \var{ins} \}
    & \text{domain restriction}
    \\
    \var{ins} \subtractdom \var{utxo}
    & = \{ i \mapsto o \mid i \mapsto o \in \var{utxo}, ~ i \notin \var{ins} \}
    & \text{domain exclusion}
    \\
    \var{utxo} \restrictrange \var{outs}
    & = \{ i \mapsto o \mid i \mapsto o \in \var{utxo}, ~ o \in \var{outs} \}
    & \text{range restriction}
  \end{align*}
  \caption{Functions used in UTxO rules}
  \label{fig:derived-defs:utxo}
\end{figure}

\begin{figure}
  \emph{UTxO transitions}
  \begin{equation*}
    \_ \vdash
    \var{\_} \trans{utxo}{\_} \var{\_}
    \subseteq \powerset (\PrtclConsts \times \UTxO \times \Tx \times \UTxO)
  \end{equation*}
  \caption{UTxO transition-system types}
  \label{fig:ts-types:utxo}
\end{figure}

\begin{figure}
  \begin{equation}\label{eq:utxo-inductive}
    \inference[UTxO-inductive]
    { \txins{tx} \subseteq \dom \var{utxo} & \minfee{pc}{tx} \leq \txfee{tx}\\
      \balance{(\txouts{tx})}  + \txfee{tx} =
        \balance{(\txins{tx} \restrictdom \var{utxo})}
    }
    {\var{pc} \vdash \var{utxo} \trans{utxo}{tx}
      (\txins{tx} \subtractdom \var{utxo}) \cup \txouts{tx}
    }
  \end{equation}
  \caption{UTxO inference rules}
  \label{fig:rules:utxo}
\end{figure}

Rule~\ref{eq:utxo-inductive} specifies under which conditions a transaction can
be applied to a set of unspent outputs, and how the set of unspent output
changes with a transaction:
\begin{itemize}
\item Each input spent in the transaction must be in the set of unspent
  outputs.
\item The balance of the unspent outputs in a transaction (i.e. the total
  amount paid in a transaction) must be equal or less than the amount of spent
  inputs.
\item If the above conditions hold, then the new state will not have the inputs
  spent in transaction $\var{tx}$ and it will have the new outputs in
  $\var{tx}$.
\end{itemize}

\begin{note}
  $\Coin$ is defined as a primitive type, but there is a difference
  between implementing it with $\mathbb{N}$ versus $\mathbb{Z}$.
  Since this is a pure UTxO ledger, $\mathbb{N}$ suffices.
  If, however, $\mathbb{Z}$ is used, then extra validation is required
  to ensure that all $\TxOut$ are non-negative.
  This extra condition would be added to \cref{eq:utxo-inductive}.
\end{note}

\subsection{Properties}
\label{sec:utxo-properties}

\begin{todo}
  Can we prove properties of the transition system of this section? For
  instance we might like to formalize ``double spending'' and prove that these
  rules prevent it. Do we want it?
\end{todo}

\subsection{Witnesses}
\label{sec:witnesses}

The rules for witnesses are presented in Figure~\ref{fig:rules:utxow}.
The definitions used in Rule~\ref{eq:utxo-witness-inductive} are given in
Figure~\ref{fig:defs:utxow}. Note that
Rule~\ref{eq:utxo-witness-inductive} uses the transition relation defined in
Figure~\ref{fig:rules:utxo}. The main reason for doing this is to define
the rules incrementally, modeling different aspects in isolation to keep the
rules as simple as possible. Also note that the $\trans{utxo}{}$ relation could
have been defined in terms of $\trans{utxow}{}$ (thus composing the rules in a
different order). The choice here is arbitrary.

\begin{figure}
  \emph{Abstract functions}
  %
  \begin{equation*}
    \begin{array}{r@{~\in~}lr}
      \fun{wits} & \Tx \to \powerset{(\VKey \times \Sig)}
      & \text{witnesses of a transaction}\\
      \fun{hash_{spend}} & \Addr \mapsto \Hash
      & \text{hash of a spending key in an address}\\
    \end{array}
  \end{equation*}
  \caption{Definitions used in the UTxO transition system with witnesses}
  \label{fig:defs:utxow}
\end{figure}

\begin{figure}
  \begin{align*}
    & \addr{}{} \in \UTxO \to \TxIn \mapsto \Addr & \text{address of an input}\\
    & \addr{utxo} = \{ i \mapsto a \mid i \mapsto (a, \wcard) \in \var{utxo} \} \\
    \nextdef
    & \fun{addr_h} \in \UTxO \to \TxIn \mapsto \Hash & \text{hash of an input address}\\
    & \fun{addr_h}~utxo = \{ i \mapsto h \mid i \mapsto (a, \wcard) \in \var{utxo}
      \wedge a \mapsto h \in \fun{hash_{spend}} \}
  \end{align*}
  \caption{Functions used in rules witnesses}
  \label{fig:derived-defs:utxow}
\end{figure}

\begin{figure}
  \emph{UTxO with witness transitions}
  \begin{equation*}
    \var{\_} \vdash
    \var{\_} \trans{utxow}{\_} \var{\_}
    \subseteq \powerset (\PrtclConsts \times \UTxO \times \Tx \times \UTxO)
  \end{equation*}
  \caption{UTxO with witness transition-system types}
  \label{fig:ts-types:utxow}
\end{figure}

\begin{figure}
  \begin{equation}
    \label{eq:utxo-witness-inductive}
    \inference[UTxO-wit]
    { \var{pc} \vdash \var{utxo} \trans{utxo}{tx} \var{utxo'}\\ ~ \\
      & \forall i \in \txins{tx} \cdot \exists (\var{vk}, \sigma) \in \wits{\var{tx}}
      \cdot
      \mathcal{V}_{\var{vk}}{\serialised{\txbody{tx}}}_{\sigma}
      \wedge  \fun{addr_h}~{utxo}~i = \hash{vk}\\
    }
    {\var{pc} \vdash \var{utxo} \trans{utxow}{tx} \var{utxo'}}
  \end{equation}
  \caption{UTxO with witnesses inference rules}
  \label{fig:rules:utxow}
\end{figure}

\section{Delegation}
\label{sec:delegation}
\section{Delegation}
\label{sec:delegation}

\newcommand{\DSEnv}{\type{DSEnv}}
\newcommand{\DSState}{\type{DSState}}
\newcommand{\DCert}{\type{DCert}}
\newcommand{\DState}{\type{DState}}
\newcommand{\DEState}{\type{DEState}}

An agent owning a key that can sign new blocks can delegate its signing rights
to another key by means of \textit{delegation certificates}. These certificates
are included in the ledger, and therefore also included in the body of the
blocks in the blockchain.

There are several restrictions on a certificate posted on the blockchain:
\begin{enumerate}
\item Only genesis keys can delegate.
\item Certificates must be properly signed by the delegator.
\item Any given key can delegate at most once per-epoch.
\item The epochs in certificates cannot refer to past epochs. This mechanism
  prevents replay attacks.
\item Two certificates for the same key issued in the same (current) slot
  cannot contain different epochs.
\item Certificates do not become active immediately, but they require a certain
  number of slots till they become stable in all the nodes.
\end{enumerate}
These conditions are formalized in \cref{fig:rules:delegation-scheduling}.
Rule~\ref{eq:rule:delegation-scheduling} determines when a certificate can
become ``scheduled''. The definitions used in this rules are presented in
\cref{fig:defs:delegation-scheduling}, and the types of the system induced by
$\trans{sdeleg}{\wcard}$ are presented in
\cref{fig:ts-types:delegation-scheduling}.

\begin{figure}[htb]
  \emph{Abstract types}
  \begin{equation*}
    \begin{array}{r@{~\in~}lr}
      c & \DCert & \text{delegation certificate}\\
      \var{vk_g} & \VKeyGen & \text{genesis verification key}\\
    \end{array}
  \end{equation*}

  \emph{Derived types}
  \begin{equation*}
    \begin{array}{r@{~\in~}l@{\qquad=\qquad}r@{~\in~}lr}
      \var{e} & \Epoch & n & \mathbb{N} & \text{epoch}\\
      \var{s} & \Slot & s & \mathbb{N} & \text{slot}\\
      \var{d} & \SlotCount & s & \mathbb{N} & \text{slot}                                         
    \end{array}
  \end{equation*}

  \emph{Constraints}
  \begin{align*}
    \VKeyGen \subseteq \VKey
  \end{align*}

  \emph{Abstract functions}
  \begin{equation*}
    \begin{array}{r@{~\in~}lr}
      \fun{dbody} & \DCert \to (\VKey \times \Epoch)
      & \text{body of the delegation certificate}\\
      \fun{dwit} & \DCert \to (\VKeyGen \times \Sig)
      & \text{witness for the delegation certificate}\\
      \fun{dwho} & \DCert \mapsto (\VKeyGen \times \VKey)
      & \text{who delegates to whom in the certificate}\\
      \fun{depoch} & \DCert \mapsto \Epoch
      & \text{certificate epoch}
    \end{array}
  \end{equation*}
  \caption{Delegation scheduling definitions}
  \label{fig:defs:delegation-scheduling}
\end{figure}

\begin{figure}[htb]
  \emph{Delegation scheduling environments}
  \begin{equation*}
    \DSEnv =
    \left(
      \begin{array}{r@{~\in~}lr}
        \mathcal{K} & \powerset{\VKeyGen} & \text{allowed delegators}\\
        \var{e} & \Epoch & \text{epoch}\\
        \var{s} & \Slot & \text{slot}\\
        \var{d} & \SlotCount & \text{certificate liveness parameter}
      \end{array}
    \right)
  \end{equation*}

  \emph{Delegation scheduling states}
  \begin{equation*}
    \DSState
    = \left(
      \begin{array}{r@{~\in~}lr}
        \var{sds} & \seqof{(\Slot \times (\VKeyGen \times \VKey))} & \text{scheduled delegations}\\
        \var{eks} & \powerset{(\Epoch \times \VKeyGen)} & \text{key-epoch delegations}
      \end{array}
    \right)
  \end{equation*}

  \emph{Delegation scheduling transitions}
  \begin{equation*}
    \var{\_} \vdash
    \var{\_} \trans{sdeleg}{\_} \var{\_}
    \subseteq \powerset (\DSEnv \times \DSState \times \DCert \times \DSState)
  \end{equation*}
  \caption{Delegation scheduling transition-system types}
  \label{fig:ts-types:delegation-scheduling}  
\end{figure}

\begin{figure}[htb]
  \begin{equation}
    \label{eq:rule:delegation-scheduling}
    \inference
    {
      (\var{vk_s},~ \sigma) = \dwit{c}
      & \verify{vk_s}{\serialised{\dbody{c}}}{\sigma} & vk_s \in \mathcal{K}\\ ~ \\
      (\var{vk_s},~ \var{vk_d}) = \dwho{c} & e_d = \depoch{c}
      & (e_d,~ \var{vk_s}) \notin \var{eks} & e < e_d \\ ~ \\
      (s + d,~ (\var{vk_s},~ \wcard)) \notin \var{sds}\\
    }
    {
      {\begin{array}{l}
       \mathcal{K}\\
        e\\
        s\\
        d
      \end{array}}
      \vdash
      {
        \left(
          \begin{array}{l}
            \var{sds}\\
            \var{eks}
          \end{array}
        \right)
      }
      \trans{sdeleg}{c}
      {
        \left(
          \begin{array}{l}
            \var{sds}; (s + d,~ (\var{vk_s},~ \var{vk_d}))\\
            \var{eks} \cup \{(e_d,~ \var{vk_s})\}
          \end{array}
        \right)
      }
    }
  \end{equation}
  \caption{Delegation scheduling rules}
  \label{fig:rules:delegation-scheduling}
\end{figure}

Once a scheduled certificate becomes active
(see~\cref{sec:delegation-interface-rules}), the delegation map is changed by
it. \cref{fig:rules:delegation} models such change.

\begin{figure}[htb]
  \begin{align*}
    & \unionoverride \in (A \mapsto B) \to (A \mapsto B) \to (A \mapsto B)
    & \text{union override}\\
    & d_0 \unionoverride d_1 = d_1 \cup (\dom d_1 \subtractdom d_0)
  \end{align*}
  \caption{Functions used in delegation rules}
  \label{fig:funcs:delegation}
\end{figure}

\begin{figure}[htb]
  \emph{Delegation states}
  \begin{align*}
    & \DState
      = \left(
        \begin{array}{r@{~\in~}lr}
          \var{dms} & \VKeyGen \mapsto \VKey & \text{delegation map}\\
          \var{dws} & \VKeyGen \mapsto \Slot & \text{when last delegation occurred}\\
        \end{array}\right)
  \end{align*}
  \emph{Delegation transitions}
  \begin{equation*}
    \_ \vdash \_ \trans{deleg}{\_} \_ \in
    \powerset (\DState \times (\Slot \times (\VKeyGen \times \VKey)) \times \DState)
    \end{equation*}
  \caption{Delegation transition-system types}
  \label{fig:ts-types:delegation}
\end{figure}

\begin{figure}[htb]
  \begin{equation}\label{eq:rule:delegation-change}
    \inference
    {\var{vk_s} \mapsto s_p \in \var{dws} \Rightarrow s_p < s
    }
    {
      \left(
      \begin{array}{r}
        \var{dms}\\
        \var{dws}
      \end{array}
      \right)
      \trans{adeleg}{(s,~ (vk_s,~ vk_d))}
      \left(
      \begin{array}{lcl}
        \var{dms} & \unionoverride & \{\var{vk_s} \mapsto \var{vk_d}\}\\
        \var{dws} & \unionoverride & \{\var{vk_s} \mapsto s \}
      \end{array}
      \right)
    }
  \end{equation}
  \nextdef
  \begin{equation}\label{eq:rule:delegation-nop}
    \inference
    {\var{kv_s} \mapsto s_p  \in \var{dws}  \wedge s \leq s_p
    }
    {
      \left(
      \begin{array}{r}
        \var{dms}\\
        \var{dws}
      \end{array}
      \right)
      \trans{adeleg}{(s,~ (vk_s,~ \wcard))}
      \left(
      \begin{array}{lcl}
        \var{dms}\\
        \var{dws}
      \end{array}
      \right)
    }
  \end{equation}
  \caption{Delegation inference rules}
  \label{fig:rules:delegation}
\end{figure}

\clearpage

\subsection{Delegation sequences}
\label{sec:delegation-sequences}

This section presents the rules that model the effect that sequences of
delegations have on the ledger.

\begin{figure}[htb]
  \begin{equation}
    \label{eq:rule:delegation-scheduling-seq-base}
    \inference
    {
    }
    {
      {\begin{array}{l}
         \mathcal{K} \\
         e\\
         s\\
         d
       \end{array}}
      \vdash
      {
        \left(
          \begin{array}{l}
            \var{sds}\\
            \var{eks}
          \end{array}
        \right)
      }
      \trans{sdelegs}{\epsilon}
      {
        \left(
          \begin{array}{l}
            \var{sds}\\
            \var{eks}
          \end{array}
        \right)
      }
    }
  \end{equation}
  \nextdef
  \begin{equation}
    \label{eq:rule:delegation-scheduling-seq-ind}
    \inference
    {
      {\begin{array}{l}
         \mathcal{K} \\
         e\\
         s\\
         d
       \end{array}}
      \vdash
      {
        \left(
          \begin{array}{l}
            \var{sds}\\
            \var{eks}
          \end{array}
        \right)
      }
      \trans{sdelegs}{\Gamma}
      {
        \left(
          \begin{array}{l}
            \var{sds'}\\
            \var{eks'}
          \end{array}
        \right)
      }
      &
      {\begin{array}{l}
         \mathcal{K} \\
         e\\
         s\\
         d
       \end{array}}
      \vdash
      {
        \left(
          \begin{array}{l}
            \var{sds'}\\
            \var{eks'}
          \end{array}
        \right)
      }
      \trans{sdeleg}{c}
      {
        \left(
          \begin{array}{l}
            \var{sds''}\\
            \var{eks''}
          \end{array}
        \right)
      }
    }
    {
      {\begin{array}{l}
         \mathcal{K} \\
         e\\
         s\\
         d
       \end{array}}
      \vdash
      {
        \left(
          \begin{array}{l}
            \var{sds}\\
            \var{eks}
          \end{array}
        \right)
      }
      \trans{sdelegs}{\Gamma; c}
      {
        \left(
          \begin{array}{l}
            \var{sds''}\\
            \var{eks''}
          \end{array}
        \right)
      }
    }
  \end{equation}
  \caption{Delegation scheduling sequence rules}
  \label{fig:rules:delegation-scheduling-seq}
\end{figure}

\begin{figure}
  \begin{equation}
    \label{eq:rule:delegation-seq-base}
    \inference
    {
    }
    {
      {
        \left(
          \begin{array}{l}
            \var{dms}\\
            \var{dws}
          \end{array}
        \right)
      }
      \trans{adelegs}{\epsilon}
      {
        \left(
          \begin{array}{l}
            \var{dms}\\
            \var{dws}
          \end{array}
        \right)
      }
    }
  \end{equation}
  \nextdef
  \begin{equation}
    \label{eq:rule:delegation-seq-ind}
    \inference
    {
      {
        \left(
          \begin{array}{l}
            \var{dms}\\
            \var{dws}
          \end{array}
        \right)
      }
      \trans{adelegs}{\Gamma}
      {
        \left(
          \begin{array}{l}
            \var{dms'}\\
            \var{dws'}
          \end{array}
        \right)
      }
      &
      {
        \left(
          \begin{array}{l}
            \var{dms'}\\
            \var{dws'}
          \end{array}
        \right)
      }
      \trans{adeleg}{c}
      {
        \left(
          \begin{array}{l}
            \var{dms''}\\
            \var{dws''}
          \end{array}
        \right)
      }
    }
    {
      {
        \left(
          \begin{array}{l}
            \var{dms}\\
            \var{dws}
          \end{array}
        \right)
      }
      \trans{adelegs}{\Gamma; c}
      {
        \left(
          \begin{array}{l}
            \var{dms''}\\
            \var{dws''}
          \end{array}
        \right)
      }
    }
  \end{equation}
  \caption{Delegations sequence rules }
  \label{fig:rules:delegation-seq}
\end{figure}


\section{Blockchain interface}
\label{sec:blockchain-interface}
\section{Blockchain interface}
\label{sec:blockchain-interface}

\newcommand{\DIEnv}{\type{DIEnv}}
\newcommand{\DIState}{\type{DIState}}

\subsection{Delegation interface}
\label{sec:delegation-interface}

\begin{figure}[htb]
  \emph{Delegation interface environments}
  \begin{equation*}
    \DIEnv =
    \left(
      \begin{array}{r@{~\in~}lr}
        \mathcal{K} & \powerset{\VKeyGen} & \text{allowed delegators}\\
        \var{e} & \Epoch & \text{epoch}\\
        \var{s} & \Slot & \text{slot}\\
        \var{d} & \SlotCount & \text{certificate liveness parameter}
      \end{array}
    \right)
  \end{equation*}

  \emph{Delegation interface states}
  \begin{equation*}
    \DIState
    = \left(
      \begin{array}{r@{~\in~}lr}
        \var{dms} & \VKeyGen \mapsto \VKey & \text{delegation map}\\
        \var{dws} & \VKeyGen \mapsto \Slot & \text{when last delegation occurred}\\
        \var{sds} & \seqof{(\Slot \times (\VKeyGen \times \VKey))} & \text{scheduled delegations}\\
        \var{eks} & \powerset{(\Epoch \times \VKeyGen)} & \text{key-epoch delegations}
      \end{array}
    \right)
  \end{equation*}

  \emph{Delegation transitions}
  \begin{equation*}
    \_ \vdash \_ \trans{deleg}{\_} \_ \in
    \powerset (\DIEnv \times \DIState \times \seqof{\DCert} \times \DIState)
  \end{equation*}
  \caption{Delegation interface transition-system types}
  \label{fig:ts-types:delegation-interface}
\end{figure}

\subsubsection{Delegation interface rules}
\label{sec:delegation-interface-rules}

\begin{figure}[htb]
  \begin{equation}
    \label{eq:rule:delegation-interface}
    \inference
    {
      {\begin{array}{l}
         \mathcal{K} \\
         e\\
         s\\
         d
       \end{array}}
      \vdash
      {
        \left(
          \begin{array}{l}
            \var{sds}\\
            \var{eks}
          \end{array}
        \right)
      }
      \trans{sdelegs}{\Gamma}
      {
        \left(
          \begin{array}{l}
            \var{sds'}\\
            \var{eks'}
          \end{array}
        \right)
      }
      &
      {
        \left(
          \begin{array}{l}
            \var{dms}\\
            \var{dws}
          \end{array}
        \right)
      }
      \trans{adelegs}{[.., s] \restrictdom \var{sds'}}
      {
        \left(
          \begin{array}{l}
            \var{dms'}\\
            \var{dws'}
          \end{array}
        \right)
      }
    }
    {
      {\begin{array}{l}
         \mathcal{K} \\
         e\\
         s\\
         d
      \end{array}}
      \vdash
      {
        \left(
          \begin{array}{l}
            \var{dms}\\
            \var{dws}\\
            \var{sds}\\
            \var{eks}
          \end{array}
        \right)
      }
      \trans{sdeleg}{\Gamma}
      {
        \left(
          \begin{array}{l}
            \var{dms'}\\
            \var{dws'}\\
            \var{[s-d,~ s+d]} \restrictdom \var{sds'}\\
            \var{[e, ..]} \restrictdom \var{eks'}
          \end{array}
        \right)
      }
    }
  \end{equation}
  \caption{Delegation interface rules}
  \label{fig:rules:delegation-interface}
\end{figure}

\subsubsection{Delegation interface functions}
\label{sec:delegation-interface-functions}

\begin{figure}[htb]
  \begin{align*}
    & \fun{delegates} \in \DIState \to (\VKeyGen \times \VKey) \to \Bool & \text{delegation relationship}\\
    & \fun{delegates}~s~(\var{vk_s}, \var{vk_d}) = \var{vk_s} \mapsto \var{vk_d} \in (\var{dms}~s)\\
    \nextdef
    & \fun{initialDS} \in \VKeyGen \to \DIState & \text{initial delegation state}\\
    & \fun{initialDS}~\var{ks} =
      \left(
      \begin{array}{l}
        \var{ks} \mapsto \var{ks}\\
        \emptyset\\
        \epsilon\\
        \emptyset\\
      \end{array}
      \right)
  \end{align*}
  \caption{Delegation interface functions}
\end{figure}


\section{Blockchain layer}
\label{sec:blockchain-layer}
\begin{note}
  This section provides a \textbf{proposal} on how the ledger rules can be used
  to build the blockchain ones. It was mainly developed to help me
  understanding what the blockchain layer requires from the ledger layer, and
  the aspects that need to be modeled in the former. In addition, my
  expectation with this section is that we can discuss which tasks should be
  completed in order to finish a first draft of the blockchain and ledger
  specifications, so that we can move forward with the generators. This section
  was not intended as a replacement of the blockchain spec, which can be found
  in a different document.-- Damian Nadales
\end{note}

\subsection{Chain extension}
\label{sec:chain-extension}

The chain extension rule is given in Figure~\ref{fig:rules:chain-extension},
and the definitions used in this rule are presented in
Figure~\ref{fig:defs:chain-extension}. Rule~\ref{eq:rule:chain-extension}
relies on transitions $\trans{bdeleg}{}$, which specify the delegation
behavior, and $\trans{butxo}{}$ which models the evolution of unspent outputs
after applying the transitions in a block. Rules for delegation and unspent
outputs in the context of a block are given in
Sections~\ref{sec:block-delegation} and \ref{sec:block-utxo} respectively.

\begin{note}
  The protocol constants are currently modeled as a part of the
  chain extension environment. Once we adding voting
  (the mechanism which controls the protocol constants)
  to the model, it will probably make sense to move them to the
  chain extension states. The protocol constants will also be
  added to the ledger state, along with some kind of rule which
  combines voting with the UTxO rule:
  \begin{equation*}
    \inference[Voting-combine]
    {
      \var{pc} \trans{voting}{} \var{pc'} &
      \var{pc'} \vdash \var{utxo} \trans{utxo}{} \var{utxo'}\\ ~ \\
    }
    {\ldots}
  \end{equation*}
\end{note}

\begin{figure}
  \emph{Abstract types}
  \begin{equation*}
    \begin{array}{r@{~\in~}lr}
      \var{b} & \Block & \text{block}\\
      \var{s} & \SlotId & \text{slot id}\\
    \end{array}
  \end{equation*}
  \emph{Abstract functions}
  \begin{equation*}
    \begin{array}{r@{~\in~}lr}
    \fun{bwit} & \Block \to (\VKey \times \Sig) & \text{block witness}\\
      \fun{bepoch} & \Block \to \Epoch & \text{block epoch}\\
      \fun{bslot} & \Block \to \SlotId & \text{block slot id}\\
    \fun{s_0} & \SlotId  & \text{slot zero (smallest slot id)}\\
    \end{array}
  \end{equation*}
  \caption{Blockchain extension definitions}
  \label{fig:defs:chain-extension}
\end{figure}

\begin{figure}
  \emph{Chain extension environment}
  \begin{equation*}
    \CEEnv =
    \left(
      \begin{array}{r@{~\in~}lr}
        \var{\Gkeys} & \powerset{\VKeyGen} & \text{genesis keys}\\
        \var{K} & \mathbb{N} & \text{number of nodes}\\
        \var{t} & \mathbb{Q} & \text{byzantine nodes ratio}\\
        \var{d} & \Epoch & \text{delegation liveness parameter}\\
        \var{pc} & \PrtclConsts & \text{protocol constants}\\
      \end{array}
    \right)
  \end{equation*}
  \emph{Chain extension states}
  \begin{equation*}
    \CEState =
    \left(
      \begin{array}{r@{~\in~}lr}
        \beta & \seqof{\Block} & \text{blockchain}\\
        \var{utxo} & \UTxO & \text{blockchain unspent outputs}\\
        \var{dmap} & \VKeyGen \mapsto \VKey & \text{blockchain delegation map}\\
        \var{signers} & \seqof{\VKeyGen} & \text{last $K$ blockchain signers}\\
        \var{sid_c} & \SlotId & \text{current slot}\\
        \var{pdlgs} & \SlotId \mapsto \seqof{\DCert} & \text{pending delegations}\\
        \var{ekeys} & \Epoch \mapsto \powerset{\VKeyGen} & \text{keys delegated per epoch}
      \end{array}
    \right)
  \end{equation*}
  \emph{Chain extension transitions}
  \begin{equation*}
    \_ \vdash \_ \trans{chain}{\_} \_ \in
      \powerset (\CEEnv \times \CEState \times \Block \times \CEState)
  \end{equation*}
  \caption{Chain extension transition-system types}
  \label{fig:ts-types:chain-extension}
\end{figure}

\begin{figure}
  \begin{equation}
    \label{eq:rule:chain-base}
    \inference[Chain-base]
    {}
    {\left(
        \begin{array}{l}
          \epsilon\\
          \var{utxo}\\
          \var{\{ (\var{vk}, \var{vk}) \mid \var{vk} \in \Gkeys\}}\\
          \emptyset\\
          \var{s_0}\\
          \emptyset\\
          \emptyset
        \end{array}
      \right)
    }
  \end{equation}

  \begin{equation}
    \label{eq:rule:chain-extension}
    \inference[Chain-ext]
    {\var{dmap}~\var{vk_g} = vk_d & \bwit{b} = (\var{vk_d}, \sigma)
      & \bslot{b} = \var{sid_n} & \var{vk_g} \in \Gkeys\\
      \var{sid_c} < \var{sid_n} & \size{\{vk_g\} \restrictdom signers} \leq K * t &
      \verify{vk_d}{\serialised{\bbody{b}}}{\sigma} \\
      \var{signers'} =
         \{ (\var{sid}, \var{vk})
          \mid  (\var{sid}, \var{vk}) \in \var{signers} \cup \{(\var{sid_n}, vk_g)\}
          , \var{sid_n} - K \leq \var{sid} \}\\
      \var{cepoch} = \fun{bepoch}~b &
      {\begin{array}{l}
         \var{cepoch}\\
         \var{sid_c}\\
         d
       \end{array}}
      \vdash
      {
        \left(
          \begin{array}{l}
            \var{dmap}\\
            \var{pdlgs}\\
            \var{ekeys}
          \end{array}
        \right)
      }
      \trans{bdeleg}{b}
      {
        \left(
          \begin{array}{r}
            \var{dmap'}\\
            \var{pdlgs'}\\
            \var{ekeys'}
          \end{array}
        \right)
      }
      \\ ~ \\
      {
        \var{pc} \vdash
        \left(
          \begin{array}{l}
            \var{utxo}\\
          \end{array}
        \right)
      }
      \trans{butxo}{b}
      {
        \left(
          \begin{array}{r}
            \var{utxo'}\\
          \end{array}
        \right)
      }
    }
    {
      {\begin{array}{l}
         \Gkeys\\
         K\\
         t\\
         d\\
         pc
      \end{array}}
      \vdash
      {
        \left(
          \begin{array}{l}
            \beta\\
            \var{utxo}\\
            \var{dmap}\\
            \var{signers}\\
            \var{sid_c}\\
            \var{pdlgs}\\
            \var{ekeys}
          \end{array}
        \right)
      }
      \trans{chain}{b}
      {
        \left(
          \begin{array}{l}
            \beta; b\\
            \var{utxo}\\
            \var{dmap'}\\
            \var{signers'}\\
            \var{sid_n}\\
            \var{pdlgs'}\\
            \var{ekeys'}
          \end{array}
        \right)
      }
    }
  \end{equation}
  \caption{Chain extension rules}
  \label{fig:rules:chain-extension}
\end{figure}

\subsection{Block delegation}
\label{sec:block-delegation}

The rule for delegation of certificates in a block is shown in
Figure~\ref{fig:rules:block-delegation}, and the new definitions used in this
rule are presented in Figure~\ref{fig:defs:block-delegation}.
Rule~\ref{eq:rule:block-delegation} relies on an inference rule that models the
state changes after applying a sequence of delegation certificates. Such rule
is shown in Figure~\ref{fig:rules:delegation-sequence}.

\begin{figure}
  \emph{Abstract functions}
  \begin{equation*}
    \begin{array}{r@{~\in~}lr}
      \fun{bdlgs} & \Block \mapsto \seqof{\DCert} & \text{delegation certificates in the block}\\
    \end{array}
  \end{equation*}
  \caption{Block delegation definitions}
  \label{fig:defs:block-delegation}
\end{figure}

\begin{figure}
  \emph{Block delegation environments}
  \begin{equation*}
    \BDEnv =
    \left(
      \begin{array}{r@{~\in~}lr}
        \var{cepoch} & \Epoch & \text{current epoch}\\
        \var{sid_c} & \SlotId & \text{current slot id}\\
        \var{d} & \Epoch & \text{delegation liveness parameter}\\
      \end{array}
    \right)
  \end{equation*}
  \emph{Block delegation states}
  \begin{equation*}
    \BDState =
    \left(
      \begin{array}{r@{~\in~}lr}
        \var{dmap} & \VKeyGen \mapsto \VKey & \text{blockchain delegation map}\\
        \var{pdlgs} & \SlotId \mapsto \seqof{\DCert} & \text{pending delegations}\\
        \var{ekeys} & \Epoch \mapsto \powerset{\VKeyGen} & \text{keys delegated per epoch}
      \end{array}
    \right)
  \end{equation*}
  \emph{Block delegation transitions}
  \begin{equation*}
    \_ \vdash \_ \trans{bdeleg}{\_} \_ \in
      \powerset (\BDEnv \times \BDState \times \Block \times \CEState)
  \end{equation*}
  \caption{Block delegation transition-system types}
  \label{fig:ts-types:block-delegation}
\end{figure}

\begin{figure}
  \begin{equation}
    \label{eq:rule:block-delegation}
    \inference[Block-dlg]
    {
      \var{pdlgs'} = \var{pdlgs} \unionoverride \{(\bslot{b} + d) \mapsto \bdlgs{b} \}\\
      \var{spast} = \{ \var{sid} \mid \var{sid} \in \var \dom ~ \var{pdlgs'}
                                    ,~ \var{sid} \leq \var{sid_c}\}\\
      \Gamma = \fun{concat}~ [ \var{dlgs} \mid (\var{sid}, \var{dlgs}) \in \var{pdlgs}
                                          ,~ \var{sid} \leq \var{sid_c}]\\
      \var{cepoch} \vdash
      {
        \left(
          \begin{array}{l}
            \var{dmap}\\
            \var{ekeys}
          \end{array}
        \right)
      }
      \trans{delegs}{\Gamma}
      {
        \left(
          \begin{array}{r}
            \var{dmap'}\\
            \var{ekeys'}
          \end{array}
        \right)
      }
    }
    {
      \begin{array}{l}
        \var{cepoch}\\
        \var{sid_c}\\
        d
      \end{array}
      \vdash
      {
        \left(
          \begin{array}{l}
            \var{dmap}\\
            \var{pdlgs}\\
            \var{ekeys}
          \end{array}
        \right)
      }
      \trans{bdeleg}{b}
      {
        \left(
          \begin{array}{r}
            \var{dmap'}\\
            \var{\var{spast} \subtractdom pdlgs'}\\
            \var{ekeys'}
          \end{array}
        \right)
      }
    }
  \end{equation}
  \caption{Block delegation rules}
  \label{fig:rules:block-delegation}
\end{figure}

\begin{figure}
  \begin{equation}
    \inference[Seq-delg-base]
    {}
    { \var{cepoch} \vdash \left(
        \begin{array}{r}
          \var{dmap}\\
          \var{ekeys}
        \end{array}
      \right)
        \trans{delegs}{\epsilon}
      \left(
        \begin{array}{r}
          \var{dmap}\\
          \var{ekeys}
        \end{array}
      \right)
    }
    \label{eq:rule:sequence-delegation-base}
  \end{equation}

  \begin{equation}
    \inference[Seq-delg-ind]
    { \var{cepoch} \vdash
      {\left(
        \begin{array}{r}
          \var{dmap}\\
          \var{ekeys}
        \end{array}
      \right)}
      \trans{delegs}{\Gamma}
      {\left(
        \begin{array}{r}
          \var{dmap'}\\
          \var{ekeys'}
        \end{array}
      \right)}
    \\ ~ \\
    \var{cepoch} \vdash
    {\left(
        \begin{array}{r}
          \var{dmap'}\\
          \var{ekeys'}
        \end{array}
      \right)}
      \trans{delegw}{c}
      {\left(
        \begin{array}{r}
          \var{dmap''}\\
          \var{ekeys''}
        \end{array}
      \right)}
    }
    { \left(
        \begin{array}{r}
          \var{dmap}\\
          \var{ekeys}
        \end{array}
      \right)
      \trans{delegs}{\Gamma; c}
      \left(
        \begin{array}{r}
          \var{dmap''}\\
          \var{ekeys''}
        \end{array}
      \right)
    }
    \label{eq:rule:sequence-delegation-inductive}
  \end{equation}
  \caption{Delegation sequence rules}
  \label{fig:rules:delegation-sequence}
\end{figure}

\subsection{Block UTxO}
\label{sec:block-utxo}

\begin{todo}
  Block-UTxO rules will have the same structure as the rules presented in
  Section~\ref{sec:block-delegation}.
\end{todo}


\addcontentsline{toc}{section}{References}
\bibliographystyle{plainnat}
\bibliography{references}

\end{document}
