\newcommand{\UProp}{\ensuremath{\type{UProp}}}
\newcommand{\UPropId}{\ensuremath{\type{UpId}}}
\newcommand{\UPropSD}{\ensuremath{\type{UpSD}}}
\newcommand{\ProtVer}{\ensuremath{\type{ProtVer}}}
\newcommand{\ProtPm}{\ensuremath{\type{Ppm}}}
\newcommand{\Rpus}{\ensuremath{\type{Rpus}}}
\newcommand{\UPVEnv}{\ensuremath{\type{UPVEnv}}}
\newcommand{\UPVState}{\ensuremath{\type{UPVState}}}
\newcommand{\UPLEnv}{\ensuremath{\type{UPLEnv}}}
\newcommand{\UPLState}{\ensuremath{\type{UPLState}}}
\newcommand{\UPREnv}{\ensuremath{\type{UPREnv}}}
\newcommand{\UPRState}{\ensuremath{\type{UPRState}}}
\newcommand{\Vote}{\ensuremath{\type{Vote}}}
\newcommand{\VEnv}{\ensuremath{\type{VEnv}}}
\newcommand{\VState}{\ensuremath{\type{VState}}}
\newcommand{\BVREnv}{\ensuremath{\type{BVREnv}}}
\newcommand{\BVRState}{\ensuremath{\type{BVRState}}}
\newcommand{\ApName}{\ensuremath{\type{ApName}}}
\newcommand{\SWVer}{\ensuremath{\type{SWVer}}}
\newcommand{\ApVer}{\ensuremath{\type{ApVer}}}

\newcommand{\upSize}[1]{\ensuremath{\fun{upSize}~\var{#1}}}
\newcommand{\upPV}[1]{\ensuremath{\fun{upPV}~\var{#1}}}
\newcommand{\upId}[1]{\ensuremath{\fun{upId}~\var{#1}}}
\newcommand{\upSig}[1]{\ensuremath{\fun{upSig}~\var{#1}}}
\newcommand{\upSigData}[1]{\ensuremath{\fun{upSigData}~\var{#1}}}
\newcommand{\upIssuer}[1]{\ensuremath{\fun{upIssuer}~\var{#1}}}
\newcommand{\upParams}[1]{\ensuremath{\fun{upParams}~\var{#1}}}
\newcommand{\upSwVer}[1]{\ensuremath{\fun{upSwVer}~\var{#1}}}
\newcommand{\vCaster}[1]{\ensuremath{\fun{vCaster}~\var{#1}}}
\newcommand{\vPropId}[1]{\ensuremath{\fun{vPropId}~\var{#1}}}
\newcommand{\vSig}[1]{\ensuremath{\fun{vSig}~\var{#1}}}

\lstset{ frame=tb,
       , language=Haskell
       , basicstyle=\footnotesize\ttfamily,
       , keywordstyle=\color{blue!80},
       , commentstyle=\itshape\color{purple!40!black},
       , identifierstyle=\bfseries\color{green!40!black},
       , stringstyle=\color{orange},
       }

\lstMakeShortInline[columns=fixed]`

\section{Update mechanism}
\label{sec:update}

This section formalizes the update mechanism by which the protocol parameters
get updated. This formalization is a simplification of the current update
mechanism implemented in
\href{https://github.com/input-output-hk/cardano-sl/}{\texttt{cardano-sl}}, and
partially documented in:
\begin{itemize}
\item \href{https://cardanodocs.com/technical/updater/}{Updater implementation}
\item \href{https://cardanodocs.com/cardano/update-mechanism/}{Update mechanism}
\item \href{https://github.com/input-output-hk/cardano-sl/blob/develop/wallet-new/docs/updates.md}{Updates technical documentation}
\item
  \href{https://github.com/input-output-hk/cardano-sl/blob/develop/docs/block-processing/us.md}{Update
    system consensus rules}
\end{itemize}

The reason for formalizing a simplified version of the current implementation
is that research work on blockchain update mechanisms is needed before
introducing a more complex update logic. Since this specification is to be
implemented in a federated setting, some of the constraints put in place in the
current implementation are no longer relevant. Once the research work is ready,
this specification can be extended to incorporate the research results.

\subsection{Update proposals}
\label{sec:update-proposals}

\begin{figure}[htb]
  \emph{Abstract types}
  %
  \begin{equation*}
    \begin{array}{r@{~\in~}lr}
      \var{up} & \UProp & \text{update proposal}\\
      \var{p} & \ProtPm & \text{protocol parameter}\\
      \var{upd} & \type{UpdData} & \text{update data}\\
      \var{upa} & \type{UpdAttrs} & \text{update attributes}\\
      \var{an} & \ApName & \text{application name}
    \end{array}
  \end{equation*}
  %
  \emph{Derived types}
  \begin{equation*}
    \begin{array}{r@{~\in~}l@{~=~}r@{~\in~}lr}
      \var{s_n} & \Slot & n & \mathbb{N} & \text{slot number}\\
      \var{pv} & \ProtVer & (\var{maj}, \var{min}, \var{alt})
      & (\mathbb{N}, \mathbb{N}, \mathbb{N}) & \text{protocol version}\\
      \var{pps} & \PPMMap & \var{pps} & \ProtPm \mapsto \Value
                                         & \text{protocol parameters map}\\
      \var{apv} & \ApVer & n & \mathbb{N}\\
      \var{swv} & \SWVer
      & (\var{an}, \var{av}) & \ApName \times \ApVer
      & \text{software version}\\
      \var{pb} & \UPropSD
      &
        {\left(\begin{array}{r l}
                 \var{pv}\\
                 \var{pps}\\
                 \var{swv}\\
                 \var{upd}\\
                 \var{upa}\\
               \end{array}\right)}
      & {
        \left(
        \begin{array}{l}
          \ProtVer\\
          \PPMMap\\
          \type{SWVer}\\
          \type{UpdData}\\
          \type{UpdAttrs}\\
        \end{array}
                   \right)
                   }
               & \text{protocol update signed data}
    \end{array}
  \end{equation*}
  \emph{Abstract functions}
  %
  \begin{equation*}
    \begin{array}{r@{~\in~}lr}
      \fun{upIssuer} & \UProp \to \VKey & \text{update proposal issuer (delegate)}\\
      \fun{upSize} & \UProp \to \mathbb{N} & \text{update proposal size}\\
      \fun{upPV} & \UProp \to \ProtVer & \text{update proposal protocol version}\\
      \fun{upId} & \UProp \to \UPropId & \text{update proposal id}\\
      \fun{upParams} & \UProp \to \mathbb{\PPMMap}
                                           & \text{proposed parameters update}\\
      \fun{upSwVer} & \UProp \to \SWVer & \text{software-version update proposal}\\
      \fun{upSig} & \UProp \to \Sig & \text{update proposal signature}\\
      \fun{upSigdata} & \UProp \to \UPropSD & \text{update proposal signed data}\\
    \end{array}
  \end{equation*}
  \caption{Update proposals definitions}
  \label{fig:defs:update-proposals}
\end{figure}

The set of protocol parameters ($\ProtPm$) is assumed to contain the following keys, which
correspond with fields of the current `BlockVersionData` structure:
\begin{itemize}
\item Slot duration: $\var{slotDuration}$
\item Maximum block size: $\var{maxBlockSize}$
\item Maximum transaction size: $\var{maxTxSize}$
\item Maximum header size: $\var{maxHeaderSize}$
\item Maximum transaction size: $\var{maxTxSize}$
\item Maximum proposal size: $\var{maxProposalSize}$
\item Transaction fee policy: $\var{txFeePolicy}$
\item Script version: $\var{scriptVersion}$
\item Update adoption threshold: $\var{upAdptThd}$. This would correspond with
  `srInitThd` in `SoftForkRule`, if we set `srMinThd` to `srInitThd` and
  `srThdDecrement` to $0$.
\item Chain stability parameter: $\var{chainStabilityParameter}$.
\item Update proposal time-to-live: $\var{upropTTL}$.
\item Certificate liveness: $\var{certificateLiveness}$.
\end{itemize}
In addition we make use of the $\var{cfmThd}$ key, which determines the portion
of the stakeholders that have to vote for a proposal to consider it confirmed.
This key does not seem to have a corresponding field in `BlockVersionData`, so
to be consistent with the existing chain the value corresponding to
$\var{cfmThd}$ can be set to $4$, which represents the majority of the $7$
genesis keys.

\subsection{Update proposals registration}
\label{sec:update-proposals-registration}

\begin{figure}[htb]
  \emph{Update proposals validity environments}
  \begin{equation*}
    \UPVEnv =
    \left(
      \begin{array}{r@{~\in~}lr}
        \var{pv} & \ProtVer & \text{adopted (current) protocol version}\\
        \var{pps} & \PPMMap & \text{adopted protocol parameters map}\\
        \var{avs} & \ApName \mapsto (\ApVer \times \Slot)
        & \text{application versions}\\
      \end{array}
    \right)
  \end{equation*}
  %
  \emph{Update proposals validity states}
  \begin{equation*}
    \UPVState
    = \left(
      \begin{array}{r@{~\in~}lr}
        \var{rpus} & \UPropId \mapsto (\ProtVer \times \PPMMap)
        & \text{registered protocol update proposals}\\
        \var{raus} & \UPropId \mapsto (\ApName \times \ApVer)
        & \text{registered software update proposals}\\
      \end{array}
    \right)
  \end{equation*}
  %
  \emph{Update proposals validity transitions}
    \begin{equation*}
    \var{\_} \vdash
    \var{\_} \trans{upv}{\_} \var{\_}
    \subseteq \powerset (\UPVEnv \times \UPVState \times \UProp \times \UPVState)
  \end{equation*}
  \caption{Update proposals validity transition-system types}
  \label{fig:ts-types:up-validity}
\end{figure}

The rules in Figure~\ref{fig:rules:up-validity} model the validity of a proposal:
\begin{itemize}
\item if an update proposal proposes a change in the protocol version, it must
  do so in a consistent manner:
  \begin{itemize}
  \item The proposed version must be lexicographically bigger than the
    current version.
  \item The major versions of the proposed and current version must differ in
    at most one.
  \item If the proposed major version is equal to the current major
    version, then the proposed minor version must be incremented by one.
  \item If the proposed major version is larger than the current major version,
    then the proposed minor version must be zero.
  \item must be consistent with the current protocol parameters:
    \begin{itemize}
    \item the proposal size must not exceed the maximum size specified by
      the current protocol parameters,
    \item the proposed new maximum block size should be not greater than twice
      current maximum block size,
    \item the maximum transaction size must be smaller than the maximum block
      size (this requirement is \textbf{crucial} for having every transaction
      fitting in a block \footnote{TODO: we should discuss if this is enough,
        since a block might contain other data that contributes to its total
        size}), and
    \item the proposed new script version can be incremented by at most 1.
    \end{itemize}
  \item must have a unique version among the current active proposals. This
    implies that a proposal is uniquely determined by the protocol version it
    proposes.
  \end{itemize}
\item if an update proposal proposes to increase the application version
  version ($\var{av}$) for a given application ($\var{an}$), then there should
  not be an active update proposal that proposes the same update.
\end{itemize}
Note that the rules in Figure~\ref{fig:rules:up-validity} allow for an update
that does not propose changes in the protocol version, or does not propose
changes the software version. However the update proposal must contain a change
proposal in any of these two aspects.
%

Also note that we do not allow for updating the protocol parameters without
updating the protocol version. If an update in the protocol parameters does not
cause a soft-fork we might use the alt version for that purpose.

\begin{figure}[htb]
  \begin{equation}
    \label{eq:func:pv-can-follow}
    \begin{array}{r c l}
      \fun{pvCanFollow}~(\var{mj_n}, \var{mi_n}, \var{a_n})~(\var{mj_p}, \var{mi_p}, \var{a_p})
      & = & (\var{mj_p}, \var{mi_p}, \var{a_p}) < (\var{mj_n}, \var{mi_n}, \var{a_n})\\
      & \wedge & 0 \leq \var{mj_n} - \var{mj_p} \leq 1\\
      & \wedge & (\var{mj_p} = \var{mj_n} \Rightarrow \var{mi_p} + 1 = \var{mi_n}))\\
      & \wedge & (\var{mj_p} + 1 = \var{mj_n} \Rightarrow \var{mi_n} = 0)
    \end{array}
  \end{equation}
  \nextdef
  \begin{equation}
    \label{eq:func:can-update}
    \begin{array}{l}
      \fun{canUpdate}~\var{pps}~\var{up}\\
      {\begin{array}{r c l}
         & = & \var{maxProposalSize} \mapsto \var{mus} \in \var{pps}\\
         & \wedge & \upSize{up} \leq \var{mus}\\
         & \wedge & (\var{maxBlockSize} \mapsto \var{bszm_{up}} \in (\upParams{up})\\
         & & \Rightarrow \var{maxBlockSize} \mapsto \var{bszm} \in \var{pps}
             \wedge  \var{bszm_{up}} \leq 2*\var{bszm})\\
         & \wedge & (\var{maxBlockSize} \mapsto \var{bszm_{up}} \in (\upParams{up})
         \wedge \var{maxTxSize} \mapsto \var{txzm_{up}} \in (\upParams{up})\\
         & & \Rightarrow \var{txzm_{up}} < \var{bszm_{up}})\\
         & \wedge & (\var{scriptVersion} \mapsto \var{sv_{up}} \in (\upParams{up}) \\
         & & \Rightarrow \var{scriptVersion} \mapsto \var{sv} \in \var{pps}
             \wedge  0 \leq \var{sv_{up}} - \var{sv} \leq 1)
       \end{array}}
    \end{array}
  \end{equation}
  \nextdef
  \begin{equation}
    \label{eq:func:av-can-follow}
    \begin{array}{r c l}
      \fun{svCanFollow}~\var{avs}~(\var{an}, \var{av}) & =
      & (\var{an} \mapsto (\var{av_c}, \wcard) \in \var{avs}
        \Rightarrow \var{av} = \var{av_c} + 1)\\
      & \wedge & (\var{an} \notin \dom~\var{avs} \Rightarrow \var{av} = 0)
    \end{array}
  \end{equation}
  \caption{Update validity functions}
\end{figure}

\begin{figure}[htb]
  \begin{equation}
    \label{eq:rule:up-av-validity}
    \inference
    {
      (\var{an}, \var{av}) = \upSwVer{up}
      & \fun{svCanFollow}~\var{avs}~(\var{an}, \var{av})
      & (\var{an}, \var{av}) \notin \range~\var{raus}
    }
    {
      {
        \begin{array}{l}
          \var{avs}
        \end{array}
      }
      \vdash
      {
        \left(
          \begin{array}{l}
            \var{raus}
          \end{array}
        \right)
      }
      \trans{upsvv}{up}
      {
        \left(
          \begin{array}{l}
            \var{raus} \unionoverride \{ \upId{up} \mapsto (\var{an}, \var{av})\}
          \end{array}
        \right)
      }
    }
  \end{equation}
  \nextdef
    \begin{equation}
    \label{eq:rule:up-pv-validity}
    \inference
    {
      \var{pid} = \upId{up}
      & \var{nv} = \upPV{up}
      & \var{pps_n} = \upParams{up}\\
      & \fun{pvCanFollow}~\var{nv}~\var{pv}
      & \fun{canUpdate}~\var{pps}~\var{up}
      & \var{nv} \notin \dom~(\range~\var{rpus})
    }
    {
      {
        \begin{array}{l}
          \var{pv}\\
          \var{pps}
        \end{array}
      }
      \vdash
      {
        \left(
          \begin{array}{l}
            \var{rpus}
          \end{array}
        \right)
      }
      \trans{uppvv}{\var{up}}
      {
        \left(
          \begin{array}{l}
            \var{rpus} \unionoverride \{ \var{pid} \mapsto (\var{nv}, \var{pps_n}) \}
          \end{array}
        \right)
      }
    }
  \end{equation}
  \nextdef
  \begin{equation}
    \label{eq:rule:up-validity-pu-nosu}
    \inference
    {
      {
        \begin{array}{l}
          \var{pv}\\
          \var{pps}
        \end{array}
      }
      \vdash
      {
        \left(
          \begin{array}{l}
            \var{rpus}
          \end{array}
        \right)
      }
      \trans{uppvv}{\var{up}}
      {
        \left(
          \begin{array}{l}
            \var{rpus'}
          \end{array}
        \right)
      }
      &
      (\var{an}, \var{av}) = \upSwVer{up} & \var{an} \mapsto \var{av} \in \var{avs}
    }
    {
      {
        \begin{array}{l}
          \var{pv}\\
          \var{pps}\\
          \var{avs}
        \end{array}
      }
      \vdash
      {
        \left(
          \begin{array}{l}
            \var{rpus}\\
            \var{raus}
          \end{array}
        \right)
      }
      \trans{upv}{\var{up}}
      {
        \left(
          \begin{array}{l}
            \var{rpus'}\\
            \var{raus}
          \end{array}
        \right)
      }
    }
  \end{equation}
  \nextdef
  \begin{equation}
    \label{eq:rule:up-validity-nopu-no}
    \inference
    {
      \var{pv} = \upPV{up} & \upParams{up} = \emptyset &
      {
        \begin{array}{l}
          \var{avs}
        \end{array}
      }
      \vdash
      {
        \left(
          \begin{array}{l}
            \var{raus}
          \end{array}
        \right)
      }
      \trans{upsvv}{up}
      {
        \left(
          \begin{array}{l}
            \var{raus'}
          \end{array}
        \right)
      }
    }
    {
      {
        \begin{array}{l}
          \var{pv}\\
          \var{pps}\\
          \var{avs}
        \end{array}
      }
      \vdash
      {
        \left(
          \begin{array}{l}
            \var{rpus}\\
            \var{raus}
          \end{array}
        \right)
      }
      \trans{upv}{\var{up}}
      {
        \left(
          \begin{array}{l}
            \var{rpus}\\
            \var{raus'}
          \end{array}
        \right)
      }
    }
  \end{equation}
  \nextdef
  \begin{equation}
    \label{eq:rule:up-validity-pu-su}
    \inference
    {
      {
        \begin{array}{l}
          \var{pv}\\
          \var{pps}
        \end{array}
      }
      \vdash
      {
        \left(
          \begin{array}{l}
            \var{rpus}
          \end{array}
        \right)
      }
      \trans{uppvv}{\var{up}}
      {
        \left(
          \begin{array}{l}
            \var{rpus'}
          \end{array}
        \right)
      }
      &
      {
        \begin{array}{l}
          \var{avs}
        \end{array}
      }
      \vdash
      {
        \left(
          \begin{array}{l}
            \var{raus}
          \end{array}
        \right)
      }
      \trans{upsvv}{up}
      {
        \left(
          \begin{array}{l}
            \var{raus'}
          \end{array}
        \right)
      }
    }
    {
      {
        \begin{array}{l}
          \var{pv}\\
          \var{pps}\\
          \var{avs}
        \end{array}
      }
      \vdash
      {
        \left(
          \begin{array}{l}
            \var{rpus}\\
            \var{raus}
          \end{array}
        \right)
      }
      \trans{upv}{\var{up}}
      {
        \left(
          \begin{array}{l}
            \var{rpus'}\\
            \var{raus'}
          \end{array}
        \right)
      }
    }
  \end{equation}
  \caption{Update proposals validity rules}
  \label{fig:rules:up-validity}
\end{figure}

\clearpage

The rule of Figure~\ref{fig:rules:up-registration} models the registration of
an update proposal:
\begin{itemize}
\item We consider the update proposal issuers to be the delegators of the key
  ($\var{vk}$) that is associated with the proposal under consideration
  ($\var{up}$).
\item We check that the issuer of a proposal was delegated by a genesis key
  (which are in the domain of $\var{dms}$).
\item the update proposal data (see the definition of $\fun{upSigdata}$) must
  be signed by the proposal issuer.
\end{itemize}

\begin{figure}[htb]
  \emph{Update proposals registration  environments}
    \begin{equation*}
    \UPREnv =
    \left(
      \begin{array}{r@{~\in~}lr}
        \var{pv} & \ProtVer & \text{adopted (current) protocol version}\\
        \var{pps} & \PPMMap & \text{adopted protocol parameters map}\\
        \var{avs} & \ApName \mapsto (\ApVer \times \Slot)
        & \text{application versions}\\
        \var{dms} & \VKeyGen \mapsto \VKey & \text{delegation map}\\
      \end{array}
    \right)
  \end{equation*}
  %
  \emph{Update proposals registration states}
  \begin{align*}
    & \UPRState = \\
    & \left(
      \begin{array}{r@{~\in~}lr}
        \var{rpus} & \UPropId \mapsto (\ProtVer \times \PPMMap)
        & \text{registered update proposals}\\
        \var{raus} & \UPropId \mapsto (\ApName \times \ApVer)
        & \text{registered software update proposals}
      \end{array}
    \right)
  \end{align*}
  %
  \emph{Update proposals registration transitions}
  \begin{equation*}
    \var{\_} \vdash
    \var{\_} \trans{upreg}{\_} \var{\_}
    \subseteq \powerset (\UPREnv \times \UPRState \times \UProp \times \UPRState)
  \end{equation*}
  \caption{Update proposals registration transition-system types}
  \label{fig:ts-types:up-registration}
\end{figure}

\begin{figure}[htb]
  \begin{equation}
    \label{eq:rule:up-registration}
    \inference
    {
      {
        \begin{array}{l}
          \var{pv}\\
          \var{pps}\\
          \var{avs}
        \end{array}
      }
      \vdash
      {
        \left(
          \begin{array}{l}
            \var{rpus}\\
            \var{raus}\\
          \end{array}
        \right)
      }
      \trans{upv}{\var{up}}
      {
        \left(
          \begin{array}{l}
            \var{rpus'}\\
            \var{raus'}\\
          \end{array}
        \right)
      }
      &
      \var{dms} \restrictrange \{\var{vk}\} \neq \emptyset\\
      \var{vk} = \upIssuer{up} &
      \mathcal{V}_{\var{vk}}\serialised{\upSigData{up}}_{(\upSig{up})}
    }
    {
      {
        \begin{array}{l}
          \var{pv}\\
          \var{pps}\\
          \var{avs}\\
          \var{dms}
        \end{array}
      }
      \vdash
      {
        \left(
          \begin{array}{l}
            \var{rpus}\\
            \var{raus}
          \end{array}
        \right)
      }
      \trans{upreg}{\var{up}}
      {
        \left(
          \begin{array}{l}
            \var{rpus'}\\
            \var{raus'}
          \end{array}
        \right)
      }
    }
  \end{equation}
  \caption{Update registration rules}
  \label{fig:rules:up-registration}
\end{figure}

\clearpage

\subsection{Voting on update proposals}
\label{sec:voting-on-update-proposals}

\begin{figure}[htb]
  \emph{Abstract types}
  %
  \begin{equation*}
    \begin{array}{r@{~\in~}lr}
      \var{v} & \Vote & \text{vote on an update proposal}
    \end{array}
  \end{equation*}
  %
  \emph{Abstract functions}
  \begin{align*}
    & \fun{vCaster} \in \Vote \to \VKey & \text{caster of a vote}\\
    & \fun{vPropId} \in \Vote \to \UPropId & \text{proposal id that is being voted}\\
    & \fun{vSig} \in \Vote \to \Sig & \text{vote signature}
  \end{align*}
  \caption{Voting definitions}
  \label{fig:defs:voting}
\end{figure}

\begin{figure}[htb]
  \emph{Voting environments}
  \begin{align*}
    & \VEnv
      = \left(
      \begin{array}{r@{~\in~}lr}
        \var{rups} & \powerset{\UPropId}
        & \text{registered update proposals}\\
        \var{dms} & \VKeyGen \mapsto \VKey & \text{delegation map}
      \end{array}\right)
  \end{align*}
  %
  \emph{Voting states}
  \begin{align*}
    & \VState
      = \left(
      \begin{array}{r@{~\in~}lr}
        \var{vts} & \powerset{(\UPropId \times \VKeyGen)} & \text{votes}
      \end{array}\right)
  \end{align*}
  %
  \emph{Voting transitions}
    \begin{equation*}
    \_ \vdash \_ \trans{addvote}{\_} \_ \in
    \powerset (\VEnv \times \VState \times \Vote \times \VState)
    \end{equation*}
  \caption{Voting transition-system types}
  \label{fig:ts-types:voting}
\end{figure}

In Rule~\ref{eq:rule:voting}:
\begin{itemize}
\item Only genesis keys can vote on an update proposal, although votes can be
  cast by delegates of these genesis keys.
\item We count one vote per genesis key that delegated to the key that is
  casting the vote.
\item The vote must refer to a registered update proposal.
\item The proposal id must be signed by the key that is casting the vote.
\item It is possible for the same genesis key to vote multiple times for
  the same proposal, however this vote will be counted once (note that we're
  taking the union of the key-proposal-id pairs).
\end{itemize}

\begin{figure}[htb]
  \begin{equation}
    \label{eq:rule:voting}
    \inference
    {
      \var{pid} = \vPropId{v} & \var{vk} = \vCaster{v} &
      \var{vts}_{\var{pid}} =
      \{ (\var{pid}, \var{vk_s}) \mid \var{vk_s} \mapsto \var{vk} \in \var{dms} \}\\
      & \var{pid} \in \var{rups} &
      \mathcal{V}_{\var{vk}}\serialised{\var{pid}}_{(\vSig{v})}\\
    }
    {
      {
        \begin{array}{l}
          \var{rups}\\
          \var{dms}
        \end{array}
      }
      \vdash
      {
        \left(
          \begin{array}{l}
            \var{vts}
          \end{array}
        \right)
      }
      \trans{addvote}{\var{v}}
      {
        \left(
          \begin{array}{l}
            \var{vts} \cup \var{vts}_{\var{pid}}\\
          \end{array}
        \right)
      }
    }
  \end{equation}
  \caption{Update voting rules}
  \label{fig:rules:voting}
\end{figure}

\clearpage

\begin{figure}[htb]
  \emph{Vote registration environments}
  \begin{align*}
    & \VEnv
      = \left(
      \begin{array}{r@{~\in~}lr}
        \var{s_n} & \Slot & \text{current slot number}\\
        \var{pps} & \PPMMap & \text{current protocol parameters map}\\
        \var{rups} & \powerset{\UPropId}
        & \text{registered update proposals}\\
        \var{dms} & \VKeyGen \mapsto \VKey & \text{delegation map}
      \end{array}\right)
  \end{align*}
  %
  \emph{Vote registration states}
  \begin{align*}
    & \VState
      = \left(
      \begin{array}{r@{~\in~}lr}
        \var{cps} & \UPropId \mapsto \Slot & \text{confirmed proposals}\\
        \var{vts} & \powerset{(\UPropId \times \VKeyGen)} & \text{votes}
      \end{array}\right)
  \end{align*}
  %
  \emph{Vote registration transitions}
    \begin{equation*}
    \_ \vdash \_ \trans{UPVOTE}{\_} \_ \in
    \powerset (\VEnv \times \VState \times \Vote \times \VState)
    \end{equation*}
  \caption{Vote registration transition-system types}
  \label{fig:ts-types:vote-reg}
\end{figure}

The rules in Figure~\ref{fig:rules:up-vote-reg} model the registration of a vote:
\begin{itemize}
\item The vote gets added to the list set of votes per-proposal ($\var{vts}$),
  via transition $\trans{addvote}{}$.
\item If the number of votes for the proposal $v$ refers to exceeds the
  confirmation threshold and this proposal was not confirmed already, then the
  proposal gets added to the set of confirmed proposals ($\var{cps}$). The
  reason why we check that the proposal was not already confirmed, is that we
  want to keep in $\var{cps}$ the earliest block number in which the proposal
  was confirmed.
\end{itemize}

\begin{figure}[htb]
  \begin{equation}
    \label{eq:rule:up-no-confirmation}
    \inference
    {
      {
        \begin{array}{l}
          \var{rups}\\
          \var{dms}
        \end{array}
      }
      \vdash
      {
        \left(
          \begin{array}{l}
            \var{vts}
          \end{array}
        \right)
      }
      \trans{addvote}{\var{v}}
      {
        \left(
          \begin{array}{l}
            \var{vts'}
          \end{array}
        \right)
      }\\
      \var{pid} = \vPropId{v}
      & \var{cfmThd} \mapsto t \in \var{pps}
      & (\size{\{\var{pid}\} \restrictdom \var{vts'}} < t
      \vee \var{pid} \in \dom~\var{cps}
      )
    }
    {
      {
        \begin{array}{l}
          s_n\\
          \var{pps}\\
          \var{rups}\\
          \var{dms}
        \end{array}
      }
      \vdash
      {
        \left(
          \begin{array}{l}
            \var{cps}\\
            \var{vts}
          \end{array}
        \right)
      }
      \trans{upvote}{\var{v}}
      {
        \left(
          \begin{array}{l}
            \var{cps}\\
            \var{vts'}
          \end{array}
        \right)
      }
    }
  \end{equation}
  \nextdef
  \begin{equation}
    \label{eq:rule:up-vote-reg}
    \inference
    {
      {
        \begin{array}{l}
          \var{rups}\\
          \var{dms}
        \end{array}
      }
      \vdash
      {
        \left(
          \begin{array}{l}
            \var{vts}
          \end{array}
        \right)
      }
      \trans{addvote}{\var{v}}
      {
        \left(
          \begin{array}{l}
            \var{vts'}
          \end{array}
        \right)
      }\\
      \var{pid} = \vPropId{v}
      & \var{cfmThd} \mapsto t \in \var{pps}
      & t \leq \size{\{\var{pid}\} \restrictdom \var{vts'}}
      & \var{pid} \notin \dom~\var{cps}
    }
    {
      {
        \begin{array}{l}
          \var{s_n}\\
          \var{pps}\\
          \var{rups}\\
          \var{dms}
        \end{array}
      }
      \vdash
      {
        \left(
          \begin{array}{l}
            \var{cps}\\
            \var{vts}
          \end{array}
        \right)
      }
      \trans{upvote}{\var{v}}
      {
        \left(
          \begin{array}{l}
            \var{cps} \unionoverride  \{\var{pid} \mapsto s_n\} \\
            \var{vts'}
          \end{array}
        \right)
      }
    }
  \end{equation}
  \caption{Vote registration rules}
  \label{fig:rules:up-vote-reg}
\end{figure}

\clearpage

\subsection{Update-proposal endorsement}
\label{sec:proposal-endorsement}

Figure~\ref{fig:ts-types:up-end} shows the types of the transition system
associated with the registration of candidate protocol versions present in
blocks. Some clarifications are in order:
\begin{itemize}
\item The $k$ parameter is used to determined when a confirmed proposal is
  stable. Given we are in a current slot $s_n$, all update proposals confirmed
  at or before slot $s_n - 2 \cdot k$ are deemed stable.
\item For the sake of conciseness, we omit the types associated to the
  transitions $\trans{fads}{}$, since they can be inferred from the types of
  the $\trans{upend}{}$ transitions.
\end{itemize}

\begin{figure}[htb]
  \emph{Update-proposal endorsement environments}
  \begin{align*}
    & \BVREnv
      = \left(
      \begin{array}{r@{~\in~}lr}
        \var{k} & \mathbb{N} & \text{chain stability parameter}\\
        \var{s_n} & \Slot & \text{current block number}\\
        (\var{pv}, \var{pps}) & \ProtVer \times \PPMMap
                             & \text{current protocol parameters map}\\
        \var{cps} & \UPropId \mapsto \Slot & \text{confirmed proposals}\\
        \var{rpus} & \UPropId \mapsto (\ProtVer \times \PPMMap)
                             & \text{registered update proposals}\\
      \end{array}\right)
  \end{align*}
  %
  \emph{Update-proposal endorsement states}
  \begin{align*}
    & \BVRState
      = \left(
      \begin{array}{r@{~\in~}lr}
        \var{fads} & \seqof{(\Slot \times (\ProtVer \times \PPMMap))}
        & \text{future protocol-version adoptions}\\
        \var{bvs} & \powerset{(\ProtVer \times \VKeyGen)}
        & \text{endorsement-key pairs}
      \end{array}\right)
  \end{align*}
  %
  \emph{Update-proposal endorsement transitions}
    \begin{equation*}
    \_ \vdash \_ \trans{upend}{\_} \_ \in
    \powerset (\BVREnv \times \BVRState
    \times (\ProtVer \times \VKey) \times \BVRState)
    \end{equation*}
  \caption{Update-proposal endorsement transition-system types}
  \label{fig:ts-types:up-end}
\end{figure}

Rules in \cref{fig:rules:up-end} specify what happens when a block issuer
signals that it is ready to upgrade to a new protocol version, given in the
rule by $\var{bv}$:
\begin{itemize}
\item The set $\var{bvs}$, containing which block issuers are ready to adopt a
  given protocol version, is updated to reflect that the block issuer
  (identified by its verifying key $\var{vk}$) is ready to upgrade to
  $\var{bv}$. Given a pair $(\var{pv}, ~\var{vk}) \in \var{bvs}$, we say that
  (the owner of) key $\var{vk}$ endorses the (proposed) protocol version
  $\var{pv}$.
\item If there are a significant number of blocks signed with $\var{bv}$
  (condition formalized in \cref{eq:predicate:canadopt}), there is a registered
  proposal (which are contained in $\var{rpus}$) which proposes to upgrade the
  protocol to version $\var{bv}$, and this update proposal was confirmed at
  least $2 \cdot k$ slots ago (to ensure stability of the confirmation), then
  we update the sequence of future protocol-version adoptions ($\var{fads}$).
\item An element $(s_c, (\var{pv_c}, \var{pps_c})$ of $\var{fads}$ represents
  the fact that protocol version $\var{pv_c}$ got enough endorsements at slot
  $s_c$. An invariant that this sequence should maintain is that it is sorted
  in ascending order on slots and on protocol versions. This means that if we
  want to know what is the next candidate to adopt at a slot $s_k$ we only need
  to look at the last element of $[.., s_k] \restrictdom \var{fads}$. Since the
  list is sorted in ascending order on protocol versions, we know that this
  last element will contain the highest version to be adopted in the slot range
  $[.., s_k]$. The $\trans{fads}{}$ transition rules take care of maintaining
  the aforementioned invariant. If a given protocol-version $\var{bv}$ got
  enough endorsements, but there is an adoption candidate as last element of
  $\var{fads}$ with a higher version, we simply discard $\var{bv}$.
\item If a registered proposal cannot be adopted, we only register the
  endorsement.
\item If a block version does not correspond to a registered or confirmed
  proposal, we just ignore the endorsement.
\end{itemize}

\begin{figure}[htb]
  \begin{equation}
    \label{eq:predicate:canadopt}
    \begin{array}{r c l}
      \fun{canAdopt}~\var{pps}~\var{bvs}~\var{bv}
      & =
      & \var{upAdptThd} \mapsto t \in pps \wedge
        t \leq \size{\{\var{bv}\} \restrictdom \var{bvs}}\\
    \end{array}
  \end{equation}
  \caption{Update-proposal endorsement functions}
\end{figure}

\begin{figure}[htb]
  \begin{equation}
    \label{eq:rule:fads-add}
    \inference
    {
      (\wcard ; (s_c, (\var{pv_c}, \wcard)) \leteq \var{fads}
      \wedge \var{pv_c} < bv) \vee \epsilon = fads
    }
    {
      {
        \left(
          \begin{array}{l}
            \var{fads}
          \end{array}
        \right)
      }
      \trans{fads}{(s_n, (\var{bv}, \var{pps_c}))}
      {
        \left(
          \begin{array}{l}
            \var{fads}; (s_n, (\var{bv}, \var{pps_c}))
          \end{array}
        \right)
      }
    }
  \end{equation}
  %
  \nextdef
  \begin{equation}
    \label{eq:rule:fads-noop}
    \inference
    {
      \wcard ; (\wcard, (\var{pv_c}, \wcard)) \leteq \var{fads} & \var{bv} \leq \var{pv_c}
    }
    {
      {
        \left(
          \begin{array}{l}
            \var{fads}
          \end{array}
        \right)
      }
      \trans{fads}{(s_n, (\var{bv}, \var{pps_c}))}
      {
        \left(
          \begin{array}{l}
            \var{fads}
          \end{array}
        \right)
      }
    }
  \end{equation}
  \nextdef
    \begin{equation}
    \label{eq:rule:up-up-invalid}
    \inference
    {
      \var{pid} \mapsto (\var{bv}, \wcard) \notin \var{rpus}
      \vee \var{pid} \notin \dom~(\var{cps} \restrictrange [.., s_n - 2 \cdot k])
    }
    {
      {
        \begin{array}{l}
          k\\
          s_n\\
          (\var{pv}, \var{pps})\\
          \var{cps}\\
          \var{rpus}
        \end{array}
      }
      \vdash
      {
        \left(
          \begin{array}{l}
            \var{fads}\\
            \var{bvs}
          \end{array}
        \right)
      }
      \trans{upend}{(\var{bv}, \var{vk})}
      {
        \left(
          \begin{array}{l}
            \var{fads}\\
            \var{bvs}
          \end{array}
        \right)
      }
    }
  \end{equation}
  \nextdef
  %
  \begin{equation}
    \label{eq:rule:up-cant-adopt}
    \inference
    {
      \var{bvs'} \leteq \var{bvs} \cup \{(\var{bv}, \var{vk})\}
      & \neg (\fun{canAdopt}~\var{pps}~\var{bvs'}~\var{bv}) \\
      \var{pid} \mapsto (\var{bv}, \wcard) \in \var{rpus}
      & \var{pid} \in \dom~(\var{cps} \restrictrange [.., s_n - 2 \cdot k])
    }
    {
      {
        \begin{array}{l}
          k\\
          s_n\\
          (\var{pv}, \var{pps})\\
          \var{cps}\\
          \var{rpus}
        \end{array}
      }
      \vdash
      {
        \left(
          \begin{array}{l}
            \var{fads}\\
            \var{bvs}
          \end{array}
        \right)
      }
      \trans{upend}{(\var{bv}, \var{vk})}
      {
        \left(
          \begin{array}{l}
            \var{fads}\\
            \var{bvs'}
          \end{array}
        \right)
      }
    }
  \end{equation}
  %
  \nextdef
  %
  \begin{equation}
    \label{eq:rule:up-canadopt}
    \inference
    {
      \var{bvs'} = \var{bvs} \cup \{(\var{bv}, \var{vk})\}
      & \fun{canAdopt}~\var{pps}~\var{bvs'}~\var{bv} \\
      \var{pid} \mapsto (\var{bv}, \var{pps_c}) \in \var{rpus}
      & \var{pid} \in \dom~(\var{cps} \restrictrange [.., s_n - 2 \cdot k])\\
      (\var{fads}) \trans{fads}{(s_n, (\var{bv}, \var{pps_c}))} (\var{fads'})
    }
    {
      {
        \begin{array}{l}
          k\\
          s_n\\
          (\var{pv}, \var{pps})\\
          \var{dms}\\
          \var{cps}\\
          \var{rpus}
        \end{array}
      }
      \vdash
      {
        \left(
          \begin{array}{l}
            \var{fads}\\
            \var{bvs}
          \end{array}
        \right)
      }
      \trans{upend}{(\var{bv}, \var{vk})}
      {
        \left(
          \begin{array}{l}
            \var{fads'}\\
            \var{bvs'}
          \end{array}
        \right)
      }
    }
  \end{equation}
  \caption{Update-proposal endorsement rules}
  \label{fig:rules:up-end}
\end{figure}

\clearpage

\subsection{Deviations from the actual implementation}
\label{sec:deviation-actual-impl}

The current specification of the voting mechanism deviates from the actual
implementation, although it should be backwards compatible with the latter.
These deviations are required to simplify the voting and update mechanism
removing unnecessary features for a simplified setting, which will use the OBFT
consensus protocol with federated genesis key holders. This in turn, enables us
to remove any accidental complexity that might have been introduced in the
current implementation. The following subsections highlight the differences
between the this specification and the current implementation.

\subsubsection{Positive votes}
\label{sec:only-positive-votes}

Genesis keys can only vote (positively) for an update proposal. In the current
implementation stakeholders can vote for or against a proposal, which makes the
voting logic more complex:
\begin{itemize}
\item there are more cases to consider
\item the current voting validation rules allow voters to change their minds
  (by flipping their vote) at most once, which requires to keep track how a
  stake holder voted and how many times. Contrast this with
  Rule~\ref{eq:rule:voting} where we only need to keep track of the set of
  key-proposal-id's pairs.
\end{itemize}

\subsubsection{Alternative version numbers}
\label{sec:alt-version-numbers-constraints}

Alternative version numbers are only lexicographically constrained. The current
implementation seems to be dependent on the order in which the update proposals
arrive: given a new update proposal $\var{up}$, if a set $X$ of update
proposals with the same minor and major versions than $\var{up}$ exist, then
the alternative version of $\var{up}$ has to be one more than the maximum
alternative number of $X$. Not only this logic seems to be brittle since it
depends on the order of arrival of the update proposals, but it requires a more
complex check (which depends on state) to determine if a proposed version can
follow the current one. By being more lenient on the alternative versions of
update proposals we can simplify the version checking logic considerably.

\subsubsection{No implicit agreement}
\label{sec:no-implicit-agreement}

We do not model the implicit agreement rule. If a proposal does not get enough
votes before the end of the voting period, then we simply discard it. At the
moment it is not clear whether the implicit agreement rule is needed.

\subsubsection{Adoption threshold}
\label{sec:adoption-threshold}

The current implementation adopts a proposal with version $\var{pv}$ if the
portion of block issuers' stakes, which issued blocks with this version, is
greater than the threshold given by:

\begin{lstlisting}
max spMinThd (spInitThd - (t - s) * spThdDecrement)
\end{lstlisting}

where:
\begin{itemize}
\item \lstinline{spMinThd} is a minimum threshold required for adoption.
\item \lstinline{spInitThd} is an initial threshold.
\item \lstinline{spThdDecrement} is the decrement constant of the initial
  threshold.
\end{itemize}

In this specification we only make use of a minimum adoption threshold,
represented by the protocol parameter $\var{upAdptThd}$ until it becomes clear
why a dynamic alternative is needed.

\subsubsection{No checks on unlock-stake-epoch parameter}
\label{sec:no-unlock-stake-epoch-check}

The rule of Figure~\ref{eq:rule:up-pv-validity} does not check the
\lstinline{bvdUnlockStakeEpoch} parameter, since it will have a different
meaning in the handover phase: its use will be reserved for unlocking the
Ouroboros-BFT logic in the software.

\subsubsection{No grouping of proposals per-application name}
\label{sec:no-app-up-grouping}

It is unclear at this moment whether we need to model the applications names
and versions, so this aspect is not modeled in the present rules.

\subsubsection{Ignored attributes of proposals}

In Figure~\ref{fig:defs:update-proposals} the types
$\type{UpdData}$, and $\type{UpdAttrs}$ are only needed to model the fact that
an update proposal must sign such data, however, we do not use them for any
other purpose in this formalization.

\subsubsection{No limits on update proposals per-key per-epoch}
\label{sec:no-up-limits}

In the current system a given genesis key can submit only one proposal per
epoch. At the moment, it is not clear what are the advantages of such
constraint:
\begin{itemize}
\item Genesis keys are controlled by the Cardano foundation.
\item Even if a genesis key falls in the hands of the adversary, only one
  update proposal can be submitted per-block, and proposals have an time to
  live of $u$ blocks. So in the worst case scenario we are looking at an
  increase in the state size of the ledger proportional to $u$.
\end{itemize}
On the other hand, having that constraint in place brings some extra complexity
in the specification, and therefore in the code that will implement it.
Furthermore, in the current system, if an error is made in an update proposal,
then if an amendment must be made within the current epoch, then a new update
proposal must be submitted with a different key, which adds extra complexity
for devops. In light of the preceding discussion, unless there is a benefit for
restricting the number of times a genesis key can submit an update proposal, we
opted for removing such constraint in the current specification.

\subsubsection{Protocol only or software only updates allowed}
\label{sec:ptonly-or-swonly-allowed}

We allow for updates in the protocol version without requiring a software
version update, and conversely, we allow for a software update without
requiring an update in the protocol.

\subsubsection{Acceptance of blocks endorsing unconfirmed proposal updates}
\label{sec:acceptance-of-uncofirmed-up-endorsements}

A consequence of enforcing the update rules in \cref{fig:rules:up-end} is that
a block that is endorsing an unconfirmed proposal gets accepted, although it
will not have any effect on the update mechanism. It is not clear at this stage
whether such block should be rejected, therefore we have chosen to be lenient.

\subsection{Questions}
\label{sec:up-questions}

This temporary section contains unanswered questions about the formalization of
update mechanism:

\begin{itemize}
\item Do we need to model the $\var{scriptVersion}$?
\item Do we allow an update proposal that proposes no update?
\end{itemize}
