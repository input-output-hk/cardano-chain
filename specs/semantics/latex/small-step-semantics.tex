\documentclass[11pt,a4paper]{article}
\usepackage[margin=2.5cm]{geometry}
\usepackage{microtype}
\usepackage{mathpazo} % nice fonts
\usepackage{amsmath}
\usepackage{amssymb}
\usepackage{amsthm}
\usepackage{latexsym}
\usepackage{mathtools}
\usepackage{stmaryrd}
\usepackage{extarrows}
\usepackage{slashed}
\usepackage[colon]{natbib}
\usepackage[unicode=true,pdftex,pdfa]{hyperref}
\usepackage{xcolor}
\usepackage{xparse}
\usepackage[capitalise,noabbrev,nameinlink]{cleveref}
\hypersetup{
  pdftitle={Small Step Semantics for Cardano},
  breaklinks=true,
  bookmarks=true,
  colorlinks=false,
  linkcolor={blue},
  citecolor={blue},
  urlcolor={blue},
  linkbordercolor={white},
  citebordercolor={white},
  urlbordercolor={white}
}
\usepackage{tikz-cd}
\usepackage{float}
\floatstyle{boxed}
\restylefloat{figure}
% For notes containing warnings, questions, etc.
\usepackage[tikz]{bclogo}
\newenvironment{question}
  {\begin{bclogo}[logo=\bcquestion, couleur=orange!10, arrondi=0.2]{ QUESTION}}
  {\end{bclogo}}
\newenvironment{todo}
  {\begin{bclogo}[logo=\bcoutil, couleur=red!5, couleurBarre=red, arrondi=0.2]{ TODO}}
  {\end{bclogo}}
%%
%% Package `semantic` can be used for writing inference rules.
%%
\usepackage{semantic}
%% Setup for the semantic package
\setpremisesspace{20pt}

\DeclareMathOperator{\dom}{dom}
\DeclareMathOperator{\range}{range}

%%
%% TODO: we should package this
%%
\newcommand{\powerset}[1]{\mathbb{P}~#1}
\newcommand{\restrictdom}{\lhd}
\newcommand{\subtractdom}{\mathbin{\slashed{\restrictdom}}}
\newcommand{\restrictrange}{\rhd}
\newcommand{\union}{\cup}
\newcommand{\unionoverride}{\mathbin{\underrightarrow\cup}}
\newcommand{\uniondistinct}{\uplus}
\newcommand{\var}[1]{\mathit{#1}}
\newcommand{\fun}[1]{\mathsf{#1}}
\newcommand{\type}[1]{\mathsf{#1}}
\newcommand{\signed}[2]{\llbracket #1 \rrbracket_{#2}}
\newcommand{\size}[1]{\left| #1 \right|}
\newcommand{\trans}[2]{\xlongrightarrow[\textsc{#1}]{#2}}
\newcommand{\seqof}[1]{#1^{*}}
\newcommand{\nextdef}{\\[1em]}
\newcommand{\bytes}{\textbf{Bytes}}

\NewDocumentCommand{\tion}{o m m m m}{
  \IfNoValueF{#1}{\var{#1}\vdash} \var{#3} \trans{#2}{#4} \var{#5}
}

\theoremstyle{definition}
\newtheorem{definition}{Definition}[section]

\theoremstyle{remark}
\newtheorem{remark}{Remark}[section]

\begin{document}
\title{Small Step Semantics for Cardano}
\author{}
\date{October 17, 2018}

\maketitle

\begin{abstract}
We define the basis for our version of ``small-step semantics'' as applied to
defining components of the Cardano cryptocurrency.
\end{abstract}

\section{Introduction}
\label{sec:introduction}

In defining Cardano, we are concerned with the means to construct inductive
datatypes satisfying some validity conditions. For example, we wish to consider
when a sequence of transactions forms a valid \textit{ledger}, or a sequence of
blocks forms a valid \textit{chain}.

This document describes the semi-formal methods we use to describe such validity
conditions and how they result in the construction of valid states.

\section{Preliminaries and Notation}

In this section, we define some preliminaries and some notation common to
Cardano specifications.

\begin{description}
\item [Functions and partial functions] We denote the set of functions from $A$
  to $B$ as $A \to B$, and a function from $A$ to $B$ as $f \in A \to B$ or $f :
  A \to B$. We
  denote the set of partial functions from $A$ to $B$ as $A \mapsto B$, and an
  equivalent partial function as $g \in A \mapsto B$ or $g : A \mapsto B$. \\
  Analogously to Haskell, we use spacing for function application. So $f ~ a$
  represents the application of $a$ to the function $f$.

\end{description}


\section{Transition Systems}
\label{sec:preliminaries}

In describing the semantics for a system $L$ we are principally concerned with
five things:

\begin{description}
\item [States] The underlying datatype of our system, whose validity we are
  required to prove.
\item [Transitions] The means by which we might move from one (valid) state to
  another.
\item [Signals] The means by which transitions are triggered.
\item [Rules] Formulae describing how to derive valid states or transitions. A
  rule has an \textit{anteccedent} (sometimes called \textit{premise}) and a
  \textit{consequent} (sometimes called \textit{conclusion}), such that if the
  conditions in the antecedent hold, the consequent is assumed to be a valid
  state or transition.
\item [Environment] Sometimes we may implicitly rely on some additional
  information being present in the system to evaluate the rules. An environment
  does not have to exist (or, equivalently, we may have an empty environment),
  but crucially an environment is not modified by transitions.
\end{description}

\begin{definition}[State transition system]
A \textit{state transition system} $L$ is given by a 5-tuple $(S, T, \Sigma, R, \Gamma)$
where:
\begin{description}
\item[$S$] is a set of (not necessarily valid) states.
\item[$\Sigma$] is a set of signals.
\item[$\Gamma$] is a set of environment values.
\item[$T$] is a set of (not necessarily valid) transitions. We have
  that \[T\subseteq\powerset{(\Gamma\times S\times\Sigma\times S)}\]
\item[$R$] is a set of derivation rules. A derivation rule is given by two
  parts: its antecedent is a logical formula in $S\cup\Sigma\cup\Gamma$. Its
  consequent is either:
  \begin{itemize}
  \item A state $s\in S$, or
  \item A transition $t\in T$.
  \end{itemize}
\end{description}
\end{definition}
\begin{remark}
  The above definition is somewhat redundant, since the transition set $T$
  implicitly defines the state, signal and environment sets. We use the above
  definition for clarity, since we often wish to refer to states and signals directly.
\end{remark}
\begin{definition}[Validity]
  For a transition system $(S, T, \Sigma, R, \Gamma)$ and environment $\gamma\in\Gamma$, we say that a state $s\in
  S$ is valid if either:
\begin{itemize}
\item It appears as a consequent of a rule $r\in R$ whose antecedent has no
  variables in $S\cup\Sigma$ and which is true as evaluated at $\gamma$, or
\item There exists a tuple $(s', \sigma, r, t)\in S\times\Sigma\times R \times
  T$ such that $s'$ is valid, the antecedent of $r$ is true as evaluated at $(\gamma, s',
  \sigma)$ and where $t=(\gamma, s', \sigma, s)$.
\end{itemize}
\end{definition}

\begin{figure}[h]
  \label{fig:notation}
  For a state transition system $L=(S,T,\Sigma, R, \Gamma)$, we use the
  following notation to represent transitions and rules. \\

  \textit{Transitions}
  \[ \_ \vdash \_ \trans{L}{\_} \_ \in T \]

  \textit{Rules}
  \[ \inference[Rule-name]
     {\textit{antecedent}}
     {\textit{consequent}}
     \in R
  \]
  \caption{Notation for a state transition system.}
\end{figure}
Figure \ref{fig:notation} gives the notation for transitions and rules. For
notational convenience, we allow for some additional constructions in the
antecedent:
\begin{itemize}
  \item Rather than a single logical predicate in the antecedent, we allow
    multiple predicate formulae. All predicates defined in the antecedent must be
    true; that is, the antecedent is treated as the conjunction of all predicates defined
    there.
  \item The antecedent may contain variable definitions of the form $z=f(s)$,
    where $z$ is an additional binding to be introduced. We will refer to these
    as ``let-bindings'' in analogy with their use in programming languages.
  \end{itemize}
We can give an example of such a rule:
  \[ \inference[Example]
    {P(A) & z = f ~ s & B = g(A,z)}
    {\tion[E1]{S2}{A}{s}{B}}
  \]
To read such a rule, we proceed as follows:
\begin{enumerate}
  \item Firstly, we look at the LHS of the consequent. This introduces the basic
    environment, initial state and signal.
  \item We then expand our set of variables using let bindings in the
    antecedent.
  \item We evaluate all predicates in the antecedent ($P(A)$ in the example above), which should now be
    defined in terms of $S\cup\Sigma\cup\Gamma$ and the variables introduced in
    step 2.
  \item If all predicates evaluate to true, then we may assume a valid
    transition to the RHS of the consequent.
\end{enumerate}
\section{Composition of Transition Systems}

Part of the benefit of this approach is that it allows us to define systems
which themselves rely on other state transition systems. To do this, we allow
for an additional construction in the antecedent.

\begin{itemize}
\item The antecedent may contain a transition of the form $\tion[ENV1]{SYS}{A}{s}{B}$.
    This may either act as a predicate (if $B$ is already bound) or as an
    introduction of $B$ as per the state transition rules for system $\var{SYS}$.
\end{itemize}

We then allow a construction as follows:

\[
  \inference[Example]
    {P(A) & z = f ~ s & \tion[E1]{S1}{A}{z}{B}}
    {\tion[E1]{S2}{A}{s}{B}}
\]

The transition in the antecedent acts as a predicate requiring that $A$
transitions to $B$ under the system $S1$, and, assuming $B$ is otherwise free
(has not been bound by the LHS of the consequent or by another let binding),
acts as an introduction rule for $B$ as determined by the transition rules for
system $S1$.

\section{Serialisation}
Various of our specifications will rely on a serialisation function which
encodes a variety of types to a binary format.

\begin{figure}[h]
  \label{fig:serialisation}
  \begin{align*}
    & \llbracket\_\rrbracket \in \textbf{Hask}\mapsto\_\to\bytes
    & \text{serialisation} \\
    & \llbracket\_\rrbracket_{A} \in \var{A}\to\bytes
  \end{align*}
  \caption{Serialisation functions}
\end{figure}

Figure \ref{fig:serialisation} gives the definition of a serialisation function.
We define this as a partial function since it is defined only on a subset of
Haskell types. As an abuse of notation, we drop the subscript $A$ for the
serialisation function on type $A$, since it is unambiguously determined by its
first argument.

We require our serialisation function to satisfy the following property:
\begin{definition}[Distributivity over injective functions]
  Let $s:\mathcal{C}\mapsto\_\to\var{D}$ be a function defined over a subset of
  objects in $\mathcal{C}$, such that $s_A:A\to\var{D}$ for all $A\in\dom{s}$.
  We say that $s$ \textit{distributes over injective functions} if, for all
  injective functions $f:A\to B$ where $A,B\in\dom{s}$, there exists a function
  $f_s:\var{D}\to\var{D}$ such that the following diagram commutes:
  \[
    \begin{tikzcd}
      A \arrow{r}{f} \arrow{d}{s_A} & B \arrow{d}{s_B} \\
      D \arrow{r}{f_s} & D
    \end{tikzcd}
  \]
\end{definition}

This property allows us to extract specific subparts of our serialised form. Our
specifications will rely in places on functions defined over the serialisesation
of a value. If we have a type $A=(B,C,D,E)$, for example, and we need to extract
$\llbracket\var{B}\rrbracket$ from $\llbracket\var{A}\rrbracket$, it is this
property which guarantees we may do so without rounnd-tripping through $A$,
since $\llbracket \pi_1 ~ A\rrbracket = (\pi_1)_{\llbracket\rrbracket} ~ \llbracket A
\rrbracket$ (where $\pi_1$ is the projection onto the first component of the tuple).
\end{document}
