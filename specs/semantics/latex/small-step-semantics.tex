\documentclass[11pt,a4paper]{article}
\usepackage[margin=2.5cm]{geometry}
\usepackage{microtype}
\usepackage{mathpazo} % nice fonts
\usepackage{amsmath}
\usepackage{amssymb}
\usepackage{amsthm}
\usepackage{latexsym}
\usepackage{mathtools}
\usepackage{stmaryrd}
\usepackage{extarrows}
\usepackage{slashed}
\usepackage[colon]{natbib}
\usepackage[unicode=true,pdftex,pdfa]{hyperref}
\usepackage{xcolor}
\usepackage[capitalise,noabbrev,nameinlink]{cleveref}
\hypersetup{
  pdftitle={Small Step Semantics for Cardano},
  breaklinks=true,
  bookmarks=true,
  colorlinks=false,
  linkcolor={blue},
  citecolor={blue},
  urlcolor={blue},
  linkbordercolor={white},
  citebordercolor={white},
  urlbordercolor={white}
}
\usepackage{float}
\floatstyle{boxed}
\restylefloat{figure}
% For notes containing warnings, questions, etc.
\usepackage[tikz]{bclogo}
\newenvironment{question}
  {\begin{bclogo}[logo=\bcquestion, couleur=orange!10, arrondi=0.2]{ QUESTION}}
  {\end{bclogo}}
\newenvironment{todo}
  {\begin{bclogo}[logo=\bcoutil, couleur=red!5, couleurBarre=red, arrondi=0.2]{ TODO}}
  {\end{bclogo}}
%%
%% Package `semantic` can be used for writing inference rules.
%%
\usepackage{semantic}
%% Setup for the semantic package
\setpremisesspace{20pt}

\DeclareMathOperator{\dom}{dom}
\DeclareMathOperator{\range}{range}

%%
%% TODO: we should package this
%%
\newcommand{\powerset}[1]{\mathbb{P}~#1}
\newcommand{\restrictdom}{\lhd}
\newcommand{\subtractdom}{\mathbin{\slashed{\restrictdom}}}
\newcommand{\restrictrange}{\rhd}
\newcommand{\union}{\cup}
\newcommand{\unionoverride}{\mathbin{\underrightarrow\cup}}
\newcommand{\uniondistinct}{\uplus}
\newcommand{\var}[1]{\mathit{#1}}
\newcommand{\fun}[1]{\mathsf{#1}}
\newcommand{\type}[1]{\mathsf{#1}}
\newcommand{\signed}[2]{\llbracket #1 \rrbracket_{#2}}
\newcommand{\size}[1]{\left| #1 \right|}
\newcommand{\trans}[2]{\xlongrightarrow[\textsc{#1}]{#2}}
\newcommand{\seqof}[1]{#1^{*}}
\newcommand{\nextdef}{\\[1em]}

\theoremstyle{definition}
\newtheorem{definition}{Definition}[section]

\theoremstyle{remark}
\newtheorem{remark}{Remark}[section]

\begin{document}
\title{Small Step Semantics for Cardano}
\author{}
\date{October 17, 2018}

\maketitle

\begin{abstract}
We define the basis for our version of ``small-step semantics'' as applied to
defining components of the Cardano cryptocurrency.
\end{abstract}

\section{Introduction}
\label{sec:introduction}

In defining Cardano, we are concerned with the means to construct inductive
datatypes satisfying some validity conditions. For example, we wish to consider
when a sequence of transactions forms a valid \textit{ledger}, or a sequence of
blocks forms a valid \textit{chain}.

This document describes the (loose) formalism by which we describe such validity
conditions and how they result in the construction of valid states.

\section{Preliminaries}
\label{sec:preliminaries}

In describing the semantics for a system $L$ we are principally concerned with
five things:

\begin{description}
\item [States] The underlying datatype of our system, whose validity we are
  required to prove.
\item [Transitions] The means by which we might move from one (valid) state to
  another.
\item [Signals] The means by which transitions are triggered.
\item [Rules] Formulae describing how to derive valid states or transitions. A
  rule has an \textit{anteccedent} and a \textit{consequent}, such that if the
  conditions in the antecedent hold, the consequent is assumed to be a valid
  state or transition.
\item [Environment] Sometimes we may implicitly rely on some additional
  information being present in the system to evaluate the rules.
\end{description}

\begin{definition}[State transition system]
A \textit{state transition system} $L$ is given by a 5-tuple $(S, T, \Sigma, R, \Gamma)$
where:
\begin{description}
\item[$S$] is a set of (not necessarily valid) states.
\item[$\Sigma$] is a set of signals.
\item[$\Gamma$] is a set of environment values.
\item[$T$] is a set of (not necessarily valid) transitions. We have
  that \[T\subseteq\powerset{(\Gamma\times S\times\Sigma\times S)}\]
\item[$R$] is a set of derivation rules. A derivation rules is given by two
  parts: its antecedent is a logical formula in $S\cup\Sigma\cup\Gamma$. Its
  consequent is either:
  \begin{itemize}
  \item A state $s\in S$, or
  \item A transition $t\in T$.
  \end{itemize}
\end{description}
\end{definition}
\begin{remark}
  The above definition is somewhat redundant, since the transition set $T$
  implicitly defines the state, signal and environment sets. We use the above
  definition for clarity, since we often wish to refer to states and signals directly.
\end{remark}
\begin{definition}[Validity]
  For a transition system $(S, T, \Sigma, R, \Gamma)$ and environment $\gamma\in\Gamma$, we say that a state $s\in
  S$ is valid if either:
\begin{itemize}
\item It appears as a consequent of a rule $r\in R$ whose antecedent has no
  variables in $S\cup\Sigma$ and which is true as evaluated at $\gamma$, or
\item There exists a tuple $(s', \sigma, r, t)\in S\times\Sigma\times R \times
  T$ such that $s'$ is valid, the antecedent of $r$ is true as evaluated at $(\gamma, s',
  \sigma)$ and where $t=(\gamma, s', \sigma, s)$.
\end{itemize}
\end{definition}

\begin{figure}[h]
  \label{fig:notation}
  For a state transition system $L=(S,T,\Sigma, R, \Gamma)$, we use the
  following notation to represent transitions and rules. \\

  \textit{Transitions}
  \[ \_ \vdash \_ \trans{L}{\_} \_ \in T \]

  \textit{Rules}
  \[ \inference[Rule-name]
     {\textit{antecedent}}
     {\textit{consequent}}
     \in R
  \]
\end{figure}
\section{Composition of Transition Systems}

Part of the benefit of this approach is that it allows us to define systems
which themselves rely on other state transition systems. In this section we
give an example of how this works.
\end{document}